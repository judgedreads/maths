\documentclass[a4paper,10pt]{article}
\usepackage{mystyle}

\begin{document}

\begin{thm}[Cosine Theorem]
	Given a triangle with vertices $A,B,C$ and respective opposing edges
	$a,b,c$, and edge $b$ along the $x$-axis:
	\[ a^2 = b^2 + c^2 - 2bc\cos\theta \]
\end{thm}

\begin{proof}
	\[
	a = |\vec{BC}| = |-\vec{AC} + \vec{AB}|
	= \left|\begin{pmatrix}-b\\0\end{pmatrix} +
		c\begin{pmatrix}\cos\theta\\\sin\theta\end{pmatrix}\right|
	\]
	\[
	\tf a^2 = (c\cos\theta - b)^2 + c^2\sin^2\theta
	= b^2 + c^2 - 2bc\cos\theta
	\]
\end{proof}

A plane is described by a point and two linearly independent vectors i.e.
\[\mathbf{r}(\lambda, \mu) = \mathbf{a} + \lambda \mathbf{u} + \mu \mathbf{v}\]

A plane can also be expressed in Cartesian form i.e.
\[ax + by + cz = d\]

\begin{ex}
	\[
		\mathbf{r} =
		\begin{pmatrix}
			1\\2\\3
		\end{pmatrix}
		+ \lambda
		\begin{pmatrix}
			1\\1\\1
		\end{pmatrix}
		+ \mu
		\begin{pmatrix}
			2\\1\\0
		\end{pmatrix}
		\, \lambda, \mu \in \RR
	\]
	\[ x = 1 + \lambda + 2\mu \]
	\[ y = 2 + \lambda + \mu \]
	\[ z = 3 + \lambda \]

	\[ \tf \lambda = z - 3, \mu = y - z + 1 \]
	\[ \tf x = 2y - z \tf x - 2y + z = 0 \]
\end{ex}

\begin{ex}[Alternative approach]
	Given a plane $P$ and two points on the plane described from the origin
	as $\mathbf{r}$ and $\mathbf{a}$, if $\mathbf{n}$ is normal to $P$ we
	have:
	\[ (\mathbf{r} - \mathbf{a}) \cdot \mathbf{n} = 0 \]
	\[ \tf \mathbf{r} \cdot \mathbf{n} = \mathbf{a} \cdot \mathbf{n} \]

	We can calculate the normal vector using the cross product:
	\[
		\begin{pmatrix}
			1\\1\\1
		\end{pmatrix}
		\wedge
		\begin{pmatrix}
			2\\1\\0
		\end{pmatrix}
		=
		\begin{pmatrix}
			-1\\2\\-1
		\end{pmatrix}
	\]
	\[ \tf -x + 2y - z = -1 + 4 - 3 \tf x - 2y + z = 0 \]
\end{ex}

% Put this in curves and surfaces?
If $r(t)$ is a parameterisation of a curve, then its length can be calculated
as $ \int_a^b | r'(t) | dt $.

For a surface $z = f(x,y)$, $\mathbf{r} = x\mathbf{i} + y\mathbf{j} +
f(x,y)\mathbf{k}$. This doesn't work for closed surfaces.

% TODO: split into sections

Partial Differentiation

We can find the slope on a surface in different directions by fixing one value.
For example if $z = xy$ and we fix $y$, then $\frac{\partial z}{\partial x} =
y$. We can fix $x$ to obtain a similar result.

\begin{ex}
	\[ f(x,y) = e^{x^2y} \]
	\[ \frac{\partial f}{\partial x} = 2xy e^{x^2y} \]
	\[ \frac{\partial f}{\partial y} = x^2 e^{x^2y} \]
\end{ex}

\begin{ex}
	\[ f(x,y) = x^2 - xy + y^2 \]
	\[ \frac{\partial f}{\partial x} = 2x - y \]
	\[ \frac{\partial ^2 f}{\partial y \partial x} = -1 \]
\end{ex}

\begin{thm}[Chain Rule]
	\[ \frac{d}{dx} f(u(x)) = \frac{df}{du} \frac{du}{dx} \]
\end{thm}

\begin{proof}
	\[ \frac{u(x+h) - u(x)}{h} \to u'(x) as h \to 0 \]
	\[ let v = \frac{u(x+h) - u(x)}{h} - u'(x), v \to 0 as h \to 0 \]
	\[ \implies u(x+h) = u(x) + (u'(x) + v)h \]
	\[ let w = \frac{f(u+k) - f(u)}{k} - f'(u), w \to 0 as k \to 0 \]
	\[ \implies f(u+k) = f(u) + (f'(u) + w)k \]
	\[ \frac{df}{dx} = \frac{f(u(x+h) - f(u(x))}{h} \]
	\[ f(u(x+h) - f(u(x))= f(u(x) + (u'(x)+v)h) - f(u(x))  = (f'(u(x) +
	w)(u'(x) + v), \]
	where $k = (u'(x)+v)h$, $k \to 0$ as $h \to 0$
	\[ \implies \lim_{h \to 0} \frac{f(u(x+h)) - f(u(x))}{h} =
	f'(u(x))u'(x) \]

\end{proof}

\begin{lemma}
	If $z(t) = (x(t), y(t))$ then
	\[ \frac{dz}{dt} = \frac{\partial z}{\partial x}\frac{dx}{dt} +
	\frac{\partial z}{\partial y}\frac{dy}{dt}. \]
\end{lemma}

\begin{proof}
	\[ \frac{dz}{dt} = \frac{z(x(t+h), y(t+h)) - z(x(t), y(t))}{h} \]
	\[ = \frac{z(x(t+h), y(t+h)) - z(x(t), y(t+h))}{h} + \frac{z(x(t),
	y(t+h)) - z(x(t), y(t))}{h} \]

	We now have two differences, one with $y$ fixed and the other with
	$x$ fixed. Applying the chain rule to each one, we obtain:

	\[ \frac{dz}{dt} = \frac{\partial z}{\partial x} x'(t) + \frac{\partial
	z}{\partial y} y'(t) \]
\end{proof}

Tangents

If $\mathbf{r}(t)$ is a point in space, then the vector tangent to it is $r(t) + \lambda
\mathbf{r}'(t)$.

\begin{ex}
	\[
		\mathbf{r}(t) =
		\begin{pmatrix}
			t^2 \\ t^3 \\ t^4
		\end{pmatrix}
		, t \in \RR
	\]
	\[
		\mathbf{r}'(t) =
		\begin{pmatrix}
			2t \\ 3t^2 \\ 4t^3
		\end{pmatrix}
	\]

	At $t=2$ we have:

	\[
		\begin{pmatrix}
			4 \\ 8 \\ 16
		\end{pmatrix}
		+ \lambda
		\begin{pmatrix}
			4 \\ 12 \\ 32
		\end{pmatrix}
		, \lambda \in \RR
	\]
\end{ex}

Similarly, a tangent plane can be described with a vector to the tangent point
and the two partial derivatives.

\[ \mathbf{p} = \mathbf{r}(x_0, y_0)
+ \lambda \frac{\partial \mathbf{r}}{\partial x}(x_0, y_0)
+ \lambda \frac{\partial \mathbf{r}}{\partial y}(x_0, y_0)
, \lambda, \mu \in \RR \]

\begin{ex}
	Find the plane tangent to $x^2 + y^2 + z^2 = 9$ at $(1,2,2)$.

	\[ \mathbf{r} = (x,y,z), z = (9 - x^2 - y^2)^{1/2} \]

	\[ \frac{\partial \mathbf{r}}{\partial x} = (1, 0, \partial z /
	\partial x) \]
	\[ \frac{\partial \mathbf{r}}{\partial y} = (0, 1, \partial z /
	\partial y) \]

	\[ \frac{\partial z}{\partial x} = - \frac{x}{(9-x^2-y^2)^{1/2}} \]
	\[ \frac{\partial z}{\partial y} = - \frac{y}{(9-x^2-y^2)^{1/2}} \]

	Plugging in the values $x=1$ and $y=2$ we obtain:

	\[ \mathbf{p} = (1,2,2) + \lambda (1, 0, -1/2) + \mu (0, 1, -1) \]
\end{ex}

\begin{ex}
	Find the plane tangent to $z = 1 - 1/4 x^2 - y^2$ at $(1,1/2,1/2)$.

	\[ \frac{\partial z}{\partial x} = -1/2 x \]
	\[ \frac{\partial z}{\partial y} = -2y \]

	Plugging in the values $x=1$ and $y=1/2$ we obtain:

	\[ \mathbf{p} = (1,1/2,1/2) + \lambda (1, 0, -1/2) + \mu (0, 1, -1) \]
\end{ex}

Directional Derivatives

Let $\mathbf{r} = \left(x_0 + tn_1, y_0 + tn_2, z(x_0 + tn_1, y_0 +
tn_2)\right)$ describe a curve in the plane, where $\mathbf{n} = (n_1, n_2)$ is
the direction vector and $t \in \RR$.

\[ \frac{d\mathbf{r}}{dt} = \left(n_1, n_2, n_1 \frac{\partial z}{\partial x}
(x_0 + tn_1, y_0 + tn_2) + n_2 \frac{\partial z}{\partial x} (x_0 + tn_1, y_0 +
tn_2)\right) \]

Given that the gradient of a curve can be expressed as $\frac{z}{\sqrt{x^2 +
y^2}}$,

\[ \frac{\frac{\partial z}{\partial x} n_1 + \frac{\partial z}{\partial y}
n_2}{\sqrt{n_1^2 + n_2^2}} \]

Hence the slope on the surface $z$ in the direction of $n$ is given by:

\[ \bigtriangledown z \mathbf{\hat{n}} \]

\begin{ex}
	\[ z(x,y) = 4 - x^2 - y^2, n = (1,1) \]
	\[ \bigtriangledown z = (-2x, -2y) = (-2, -2) \]
	\[ \mathbf{\hat{n}} = \frac{\sqrt{2}}{2} (1,1) \]
	\[ \bigtriangledown z \mathbf{\hat{n}} = \frac{-4}{\sqrt{2}}
	   = -2 \sqrt{2} \]
\end{ex}

\begin{ex}
	$f(x,y,z) = e^{xyz}$ at $(2,-1,2)$ with $n = (1,1,1)$
	\[ \bigtriangledown f = (yze^{xyz}, xze^{xyz}, xye^{xyz})
		= (-2e^{-4}, 4e^{-4}, -2^{-4}) \]
	\[ \mathbf{\hat{n}} = 1/\sqrt{3}(1,1,1) \]
	\[ \bigtriangledown f \cdot \mathbf{\hat{n}} = 1/\sqrt{3}(-2e^{-4} +
	4e^{-4} - 2e^{-4}) = 0 \]
\end{ex}

Greatest Slope

\[ \bigtriangledown z \cdot \mathbf{\hat{n}} = |\bigtriangledown
z||\mathbf{\hat{n}}|\cos \theta \]

In order to maximise this, we let $\cos \theta = 1$ and given that
$\mathbf{\hat{n}}$ is a unit vector, the greatest slope is $|\bigtriangledown
z|$. Therefore, the direction of the greatest slope is in the direction of
$\bigtriangledown z$, since $|a|^2 = a \cdot a$.

\begin{ex}
	$z(x,y) = 4 - x^2 - y^2$ at $(1,1)$
	\[ \bigtriangledown z = (-2,-2) \]
	\[ |\bigtriangledown z| = 2\sqrt{2} \]
\end{ex}

Volume under a surface

Let $f(x,y)$ be a surface, the volume underneath it is
\[ v = \int_a^b \int_{x_1(y)}^{x_2(y)} f(x,y) dx dy \]

\begin{ex}
	$z = x^2y^3$ above the region bound by $y=1$, $x=4$, and $y=\sqrt{x}$.

	\[ v = \int_1^4 \int_1^{\sqrt{x}} x^2y^3 dy dx \]
	\[ = 1/4 \int_1^4 \left[ x^2y^4 \right]_1^{\sqrt{x}} \]
	\[ = 1/4 \int_1^4 x^4 - x^2 dx \]
	\[ = 1/4 \left[ 1/5 x^5 - 1/3 x^3 \right]_1^4 \]
	\[ = 1024/20 - 64/12 - 1/20 + 1/12 = 45.9 \]

	This can also be calculated by integrating first with respect to $x$.

	\[ v = \int_1^2 \int_{y^2}^4 x^2y^3 dxdy \]
	\[ = 1/3 \int_1^2 \left [x^3y^3 \right]_{y^2}^4 dy \]
	\[ = 1/3 \int_1^2 64y^3 - y^9 dy \]
	\[ = 1/3 \left[ 16y^4 - 1/10 y^10 \right]_1^2 \]
	\[ = 1/3 \left( 256 - 1024/10 - 16 + 1/10 \right) = 45.9 \]

\end{ex}

Polar Coordinates

When changing coordinate system, we have to take into account how that affects
differentiation and integration.

Let $\mathbf{r}(x,y) = x \mathbf{i} + y \mathbf{j}$. A change of $dx$ and $dy$
would give:

\[ (x + dx)\mathbf{i} + y\mathbf{j} - (x\mathbf{i} + y\mathbf{j}) =
dx\mathbf{i} \]
\[ x\mathbf{i} + (y + dy)\mathbf{j} - (x\mathbf{i} + y\mathbf{j}) =
dy\mathbf{j} \]

The differential area can be calculated using the cross product (think about
the area of a parallelogram). Given that $dx$ and $dy$ are just scalars, and
$\mathbf{i}$ and $\mathbf{j}$ are orthogonal unit vectors:

\[ |dx\mathbf{i} \times dy\mathbf{j}| = |dxdy\mathbf{k}| = dxdy \]

Now lets see how things change when we switch to polar coordinates. Let
$\mathbf{f}(r,\theta) = r\cos\theta\mathbf{i} + r\sin\theta\mathbf{j}$.

A change by $dr$ yields:
\[ (r+dr)\cos\theta\mathbf{i} + (r+dr)\sin\theta\mathbf{j} -
r\cos\theta\mathbf{i} - r\sin\theta\mathbf{j} \]
\[ = dr(\cos\theta\mathbf{i} + \sin\theta\mathbf{j} \]

A change by $d\theta$ yields:
\[ r\cos(\theta + d\theta)\mathbf{i} + r\sin(\theta + d\theta)\mathbf{j} -
r\cos\theta\mathbf{i} - r\sin\theta\mathbf{j} \]
\[ = r\left(\cos(\theta + d\theta) - \cos(\theta)\right)\mathbf{i}
+ r\left(\sin(\theta + d\theta) - \sin(\theta)\right)\mathbf{j} \]

Given that:
\[ \frac{d}{d\theta} \cos\theta = \frac{\cos(\theta + d\theta) -
\cos\theta}{d\theta} \]

We find that a change by $d\theta$ is simply:
\[ -r\sin(\theta)d\theta \mathbf{i} + r\cos(\theta)d\theta \mathbf{j} \]

Now we compute the full differential area:
\[ | dr(\cos\theta \mathbf{i} + \sin\theta \mathbf{j}) \wedge
d\theta(-r \sin\theta \mathbf{i} + r\cos\theta \mathbf{j} | \]
\[ = | r\cos^2\theta + r\sin^2\theta | drd\theta = rdrd\theta \]

\begin{ex}
	\[ I = \int \int x^2 y^2 dxdy \]
	\[ = \int_0^{\pi} \int_0^1 (r^2 \cos^2\theta)(r^2 \sin^2\theta)
	rdrd\theta \]
	\[ = \int_0^{\pi} \cos^2\theta \sin^2\theta d\theta
	\int_0^1 r^5 dr \]

	Given that $\cos2\theta = \cos^2\theta - \sin^2\theta$, we have:

	\[ \sin^2\theta = \frac{1}{2} - \frac{1}{2} \cos2\theta \]
	\[ \cos^2\theta = \frac{1}{2} + \frac{1}{2} \cos2\theta \]

	Hence:

	\[
		\sin^2\theta \cos^2\theta =
		(\frac{1}{2} - \frac{1}{2} \cos2\theta)
		(\frac{1}{2} + \frac{1}{2} \cos2\theta)
	\]
	\[
		= \frac{1}{4} - \frac{1}{4} \cos^2 2\theta
	\]
	\[
		= \frac{1}{4} - \frac{1}{4}
		(\frac{1}{2} + \frac{1}{2} \cos4\theta)
	\]
	\[
		= \frac{1}{8} - \frac{1}{8} \cos4\theta
	\]
	\[
		\int_0^{\pi} \frac{1}{8} - \frac{1}{8} \cos4\theta d\theta =
		\left[ \theta + \sin4\theta \right]_0^{\pi} = \pi/8
	\]
	\[
		\int_0^1 r^5 dr = \left[ \frac{1}{6} r^6 \right]_0^1
		= \frac{1}{6}
	\]

	Therefore,

	\[ I = \pi/48. \]

\end{ex}

We can generalise this concept of coordinate transformation for a
generic function
$ \mathbf{r} (s,t) = x(s,t)\mathbf{i} + y(s,t)\mathbf{j}$.

Changing by $ds$:
\[
	\frac{d\mathbf{r}}{ds} ds =
	\left( \frac{dx}{ds}\mathbf{i} + \frac{dy}{ds}\mathbf{j} \right) ds
\]

Changing by $dt$:
\[
	\frac{d\mathbf{r}}{dt} dt =
	\left( \frac{dx}{dt}\mathbf{i} + \frac{dy}{dt}\mathbf{j} \right) dt
\]

The resultant change in area is:
\[
	| \frac{dr}{ds}ds \wedge \frac{dr}{dt}dt |
	= | \frac{dx}{ds}\frac{dy}{dt} - \frac{dx}{dt}\frac{dy}{ds} | dsdt
\]

This leads us to the following definition.

% TODO: these should be partial derivatives (and above)
\begin{defn}[Jacobian]
	Let $x = x(s,t)$, $y = y(s,t)$, then
	$J(s,t) = |\frac{dx}{ds}\frac{dy}{dt} - \frac{dx}{dt}\frac{dy}{ds}|$
	is the Jacobian.
\end{defn}

\begin{ex}[Ellipse]
	Let $x = r\cos\theta$, $y = \frac{b}{a}r\sin\theta$. Let $\Omega$ be the
	region where $0 <= \theta <= 2\pi$, $0 <= r <= a$.

	\[
		J(r,\theta) = \left| \frac{\partial x}{\partial r}
		\frac{\partial y}{\partial \theta} -
		\frac{\partial x}{\partial \theta}
		\frac{\partial y}{\partial r} \right|
	\]
	\[
		\frac{\partial x}{\partial r} = \cos\theta
		\quad
		\frac{\partial x}{\partial \theta} = -r \sin\theta
	\]
	\[
		\frac{\partial y}{\partial r} = \frac{b}{a} \sin\theta
		\quad
		\frac{\partial y}{\partial \theta} = \frac{b}{a}r \cos\theta
	\]
	\[
		\Rightarrow J(r,\theta) = \left| \frac{b}{a}r \cos^2\theta +
		\frac{b}{a}r \sin^2\theta \right| = \frac{b}{a}r
	\]
	\[
		\Rightarrow \int\int_{\Omega} dxdy =
		\int_0^{2\pi} \int_0^a \frac{b}{a}r drd\theta
	\]
	\[
		= \int_0^{2\pi} \frac{ba}{2} d\theta = ab\pi
	\]
\end{ex}

Surface Area

Take a parallelogram with corners:

\[
	A = \begin{pmatrix}
		x \\
		y \\
		f(x,y)
	\end{pmatrix}
	B = \begin{pmatrix}
		x+dx \\
		y \\
		f(x+dx,y)
	\end{pmatrix}
	C = \begin{pmatrix}
		x \\
		y+dy \\
		f(x,y+dy)
	\end{pmatrix}
	D = \begin{pmatrix}
		x+dx \\
		y+dy \\
		f(x+dx,y+dy)
	\end{pmatrix}
\]

We then calculate the area of this using the cross product:

\[
	\vec{AB} =
	\begin{pmatrix}
		x+dx \\
		y \\
		f(x+dx,y)
	\end{pmatrix}
	-
	\begin{pmatrix}
		x \\
		y \\
		f(x,y)
	\end{pmatrix}
	=
	dx
	\begin{pmatrix}
		1 \\
		0 \\
		\partial f / \partial x
	\end{pmatrix}
\]
\[
	\vec{AC} =
	\begin{pmatrix}
		x \\
		y+dy \\
		f(x,y+dy)
	\end{pmatrix}
	-
	\begin{pmatrix}
		x \\
		y \\
		f(x,y)
	\end{pmatrix}
	=
	dy
	\begin{pmatrix}
		0 \\
		1 \\
		\partial f / \partial y
	\end{pmatrix}
\]
\[
	\vec{AB} \wedge \vec{AC} =
	dxdy
	\begin{vmatrix}
		i & j & k \\
		1 & 0 & \partial f / \partial x \\
		0 & 1 & \partial f / \partial y
	\end{vmatrix}
	=
	dxdy
	\begin{pmatrix}
		- \partial f / \partial x \\
		- \partial f / \partial y \\
		1
	\end{pmatrix}
\]
\[
	area = dxdy
	\begin{vmatrix}
		- \partial f / \partial x \\
		- \partial f / \partial y \\
		1
	\end{vmatrix}
	=
	dxdy
	\left(
		(\partial f / \partial x)^2,
		(\partial f / \partial y)^2,
		1
	\right)^{1/2}
\]

To get the full surface area, we need to sum these parallelograms. To do
this, we use integration, similarly to calculating the length of a
curve. In fact, this is essentially a two dimensional analogue.

Generalising our derivation, by using the fact that our parallelograms
are constructed by partial derivatives, we arrive at the following
formula for surface area.

Let $r(s,t) = (x(s,t), y(s,t), z(s,t))$ where $s,t \in R$, and $R$ is the
region of surface area. The surface area is calulated by:

\[
	S = \int \int_R \left| \frac{\partial r}{\partial s} \wedge
	\frac{\partial r}{\partial t} \right| dsdt
\]

Notice from our derivation of this formula that if the surface can be
expressed as $(x,y,f(x,y))$, the surface area is much simpler:

\[
	S = \int \int_R \left|
	\left(
		(\partial f / \partial x)^2,
		(\partial f / \partial y)^2,
		1
	\right)^{1/2}
	\right|
	dxdy
\]

\begin{ex}
	Let $z=xy$ and let$R$ be the cylinder defined by $x^2 + y^2 = 1$.

	\begin{align*}
		S &= \int \int_R \left( x^2 + y^2 + 1 \right)^{1/2} dxdy \\
		&= \int_0^{2\pi} \int_0^1 r(1+r^2)^{1/2} drd\theta \\
		&= \int_0^{2\pi}
		\left[
			\frac{1}{2} \frac{2}{3} (1+r^2)^{3/2}
		\right]_0^1 d\theta \\
		&= \frac{1}{3} \int_0^{2\pi} 2^{3/2} - 1 d\theta \\
		&= \frac{1}{3} \sqrt{8} 2\pi - 2\pi \\
		&= \frac{2\pi}{3} (\sqrt{8} - 1)
	\end{align*}
\end{ex}

% TODO: move this to another file
Differential Equations

\begin{defn}[Fixed Point]
	Let $f(x) = \frac{dx}{dt}$ then $x^*$ is a fixed point if $f(x^*) = 0$.
	Furthermore, fixed points can be stable or unstable:
	\begin{align*}
		f'(x^*) &> 0 \Rightarrow unstable \\
		f'(x^*) &< 0 \Rightarrow stable \\
		f'(x^*) &= 0 \Rightarrow constant
	\end{align*}
\end{defn}

Let $x^*$ be a fixed point for $f(x) = \frac{dx}{dt}$. We can express
points around $x^*$ as $x(t) = x^* + z(t)$, where $z$ is small.

Given that $\frac{d}{dt}(x^* + z(t)) = f(x^* + z(t))$ and
$\frac{d}{dt}(x^*) = 0$, we have
\[
	\frac{dz}{dt} = f(x^* + z(t))
\]

We can expand $f(x)$ around the point $x^*$ using a Taylor series:
\[
	f(x) = f(x^*) + f'(x^*)(x - x^*) + \frac{1}{2}f''(x^*)(x - x^*)^2 + \dots \\
\]
With $x = x^* + z$ we have:
\[
	f(x^* + z) = f(x^*) + f'(x^*)z + \frac{1}{2}f''(x^*)z^2 + \dots \\
\]
$z$ is small by assumption so we ignore terms with $z^2$ and onwards, yielding:
\[
	\frac{dz}{dt} = f(x^* + z) = f(x^*) + f'(x^*)z
\]
Given that $f(x^*) = 0$, we have
\[
	\frac{dz}{dt} = f'(x^*)z
\]

We now aim to find a set of such functions $z(t)$ satisfying this equation.
\[
	\frac{dz}{dt} = \lambda z, \quad \lambda = f'(x)
\]

We look for a solution of the form $Ce^{kt}$, where $C$ is constant,
since $\frac{d}{dt} e^t = e^t$.

\begin{gather*}
	z(t) = Ce^{kt} \\
	\Rightarrow z'(t) = kCe^{kt} = \lambda z = \lambda Ce^{kt} \\
	\Rightarrow k = \lambda \\
	\Rightarrow z(t) = Ce^{\lambda t} \\
	z(0) = C \\
	\Rightarrow z(t) = z(0)e^{\lambda t}
\end{gather*}

If $\lambda < 0$, $z \to 0$ as $t \to \infty$ (stable). \\
If $\lambda > 0$, $z \to \infty$ as $t \to \infty$ (unstable). \\
If $\lambda = 0$, $z$ is constant.

Linear First Orders

\[
	\frac{dy}{dx} + p(x)y = q(x)
\]

We want to get the LHS in the form
\[
	y\frac{dg}{dx} + g\frac{dy}{dx} = \frac{d}{dx} gy
\]
so that we can solve with one integral.

Mutliplying through by $g(x)$ we get:
\begin{gather*}
	g(x)\frac{dy}{dx} + g(x)p(x)y = y\frac{dg}{dx} + g\frac{dy}{dx} \\
	\Rightarrow \frac{dg}{dx} = p(x)g(x) \\
	\Rightarrow \int \frac{1}{g(x)} \frac{dg}{dx} dx = \int p(x) dx \\
	\Rightarrow \ln |g(x)| = \int p(x) dx \\
	\Rightarrow g(x) = e^{\int p(x) dx}
\end{gather*}

So now we need to solve
\[
	g(x)\frac{dy}{dx} + g(x)p(x)y = g(x)q(x)
\]
which can be simplified to
\[
	\frac{d}{dx} gy = gq
\]

Substituting in $g(x)$ we have:
\[
	y = e^{\int p} \int q e^{\int p} dx
\]

\begin{ex}
	\begin{gather*}
		\frac{dy}{dx} - \frac{y}{x^2} = - \frac{1}{x^2} \\
		\Rightarrow g(x) = \exp{\int -x^{-2}} = e^{1/x} \\
		\Rightarrow \frac{d}{dx} gy = - \frac{e^{1/x}}{x^2} \\
		\Rightarrow y e^{1/x} = e^{1/x} + c
	\end{gather*}

	Using the original equation, it is easy to see that $c=0$ and $y=1$.
\end{ex}

\begin{defn}[Homogeneous Equation]
	A differential equation is homogeneous if there are no terms
	without a derivative of $y$ (or $y$ itself), e.g.
	\[
		\frac{dy}{dx} + f(x)y = 0
	\]
\end{defn}

\begin{defn}[Eigen Function]
	An Eigen function is a function that when passed through another
	function results in multiplication, e.g.
	\[ \frac{d}{dx} e^{kx} = ke^{kx} \]
	Here $e^{kx}$ is the Eigen function that when passed through
	$\frac{d}{dx}$ results in multiplication by $k$, which is
	refered to as the Eigen value.
\end{defn}

We will use these definitions to solve linear second order O.D.Es.

\begin{ex}
	\[
		\frac{d^2y}{dx^2} + \frac{dy}{dx} - 2y = 0
	\]

	We try $y = e^{mx}$ as we are aware that it is an Eigen function
	under differentiation. Plugging in, we get:

	\begin{gather*}
		(m^2 + m - 2)e^{mx} = 0 \\
		\Rightarrow m^2 + m - 2 = 0 \\
		\Rightarrow (m+2)(m-1) = 0 \\
		\Rightarrow m = -2, 1
	\end{gather*}

	% TODO: go into detail with linear combination with Wronskian
	% etc.
	So we have two particular solutions to our equation, namely
	$e^x$ and $e^{-2x}$. With two independent solutions, we can use a
	linear combination of these to construct our general solution, much
	in the same way that we can describe a plane using two linearly
	independent vectors. Hence, the general solution to our equation is:
	\[
		y = Ae^x + Be^{-2x}
	\]
\end{ex}

\begin{ex}
	\[
		\frac{d^2y}{dx^2} + \frac{dy}{dx} + y = 0
	\]

	\begin{gather*}
		m^2 + m + 1 = 0 \\
		m = \frac{-1 \pm \sqrt{1-4}}{2} \\
		= \frac{-1 \pm \sqrt{3}i}{2}
	\end{gather*}

	Given that
	\[
		e^{i\theta} = \cos{\theta} + i\sin{\theta}
	\]
	we have
	\[
		e^{mx} = e^{-x/2}\left(\cos(\pm \frac{\sqrt{3}}{2}x)
		+ i\sin(\pm\frac{\sqrt{3}}{2}x)\right)
	\]
	and given that
	\begin{align*}
		\cos(-\theta) &= \cos(\theta) \\
		\sin(-\theta) &= -\sin(\theta)
	\end{align*}
	our particular solutions can be simplified to
	\[
		y = e^{-x/2}\left(\cos \frac{\sqrt{3}}{2}x
		\pm i\sin\frac{\sqrt{3}}{2}x\right)
	\]

	Given that we have two particular solutions, the general solution is:
	\[
		y = e^{-x/2}\left(A\cos \frac{\sqrt{3}}{2}x
		\pm B\sin\frac{\sqrt{3}}{2}x\right)
	\]
	where $A,B \in \CC$.

	In fact, if the roots of the characteristic equation are complex
	conjugates $\alpha \pm i\beta$, then the general solution can be
	expressed:
	\[
		y = Ae^{\alpha x}\cos\beta x + Be^{\alpha x}\sin\beta x
	\]

\end{ex}

\begin{ex}
	\[
		\frac{d^2y}{dx^2} + 6 \frac{dy}{dx} + 9y = 0
	\]

	\begin{gather*}
		(m^2 + 6m + 9)e^{mx} = 0 \\
		m^2 + 6m + 9 = 0 \\
		(m+3)^2  = 0 \\
		m = -3
	\end{gather*}

	Here we get repeated roots, rather than distinct, which means we
	can't form a general solution yet.

	% TODO: explain why factor of x works.
	The general form of these solutions is $(Ax+B)e^{mx}$, so our
	general solution is:
	\[
		y = (Ax+B)e^{-3x}
	\]
\end{ex}

We now look at some inhomogeneous equations.

\begin{ex}
	\[
		\frac{d^2}{dx^2} - 4\frac{dy}{dx} + 5y = 1
	\]
	We try a scalar value for $y$ since the RHS is scalar: $y =
	\frac{1}{5}$. It is easy to see that this is a particular
	solution.
\end{ex}

\begin{ex}
	\[
		\frac{d^2}{dx^2} - 4\frac{dy}{dx} + 4y = x
	\]
	We now try $y = Ax+B$ since the RHS is $x$. Plugging in we get:
	\[
		0 - 4A + 4Ax + 4B = x
	\]
	from which it is quite easy to see that $A = B = \frac{1}{4}$. Hence
	\[
		y = \frac{1}{4}x + \frac{1}{4}
	\]
\end{ex}

\begin{ex}
	\[
		\frac{d^2}{dx^2} + 4\frac{dy}{dx} + 5y = x^2
	\]
	We now try $y = Ax^2 + Bx + C$:
	\begin{gather*}
		2A + 8Ax + 4B + 5Ax^2 + 5Bx + 5C = x^2 \\
		\Rightarrow A = \frac{1}{5} \\
		\Rightarrow \frac{2}{5} + \frac{8}{5}x + 4B + 5Bx + 5C = 0 \\
		\Rightarrow 5B = -\frac{8}{5} \\
		\Rightarrow B = -\frac{8}{25} \\
		\Rightarrow \frac{2}{5} - \frac{32}{25} = -5C \\
		\Rightarrow C = \frac{22}{125} \\
		\Rightarrow y = \frac{1}{5}x^2 - \frac{8}{25} x + \frac{22}{125}
	\end{gather*}
\end{ex}

\begin{ex}
	\[
		\frac{d^2}{dx^2} - 3\frac{dy}{dx} + y = 3\cos x
	\]
	We try $y = A\sin x + B \cos x$
	\begin{gather*}
		-A\sin x - B\cos x - 3A\cos x + 3B\sin x + A\sin x + B\cos x
		= 3\cos x \\
		\Rightarrow -A + 3B + A = 0 \Rightarrow B = 0 \\
		\Rightarrow -B - 3A + B = 3 \Rightarrow A = -1 \\
		\Rightarrow y = -\sin x
	\end{gather*}
\end{ex}

To get the general solution to a second order linear inhomogeneous
equation, we need to find the general solution to the corresponding
homogeneous function, $y_c$, and the particular solution to the
inhomogeneous equation, $y_p$. Then the general solution to the
inhomogeneous equation is given by
\[
	y = y_c + y_p
\]

\begin{ex}
	\[
		\frac{d^2}{dx^2} - 4\frac{dy}{dx} + 4y = x
	\]
	As we saw before, $y_p = \frac{1}{4}x + \frac{1}{4}$.

	The homogeneous function is:
	\[
		\frac{d^2}{dx^2} - 4\frac{dy}{dx} + 4y = 0
	\]
	We solve using the same method as earlier:
	\begin{gather*}
		m^2 - 4m + 4 = 0 \\
		(m - 2)^2 = 0 \\
		m = 2 \\
		\Rightarrow y_c = (Ax+B)e^{2x} \\
	\end{gather*}
	Now we can calculate the general solution $y = y_c + y_p$:
	\[
		y = (Ax+B)e^{2x} + \frac{1}{4}x + \frac{1}{4}
	\]
\end{ex}

Variation of parameters

Let $y_p = q_1(x) v_1(x) + q_2(x) v_2(x)$, then:
\[
	y_p' = q_1v_1' + q_1'v_1 + q_2v_2' + q_2'v_2
\]
Letting $q_1'v_1 + q_2'v_2 = 0$ yields
\begin{gather*}
	y_p' = q_1v_1' + q_2v_2' \\
	\Rightarrow y_p'' = q_1'v_1' + q_1v_1'' + q_2'v_2' + q_2v_2''
\end{gather*}

Substituting into
\[
	y'' + a_1 y' + a_0 y = g(x) \\
\]
we get
\begin{multline*}
	(q_1'v_1' + q_1v_1'' + q_2'v_2' + q_2v_2'') +
	a_1(q_1v_1' + q_1'v_1 + q_2v_2' + q_2'v_2) + \\
	a_0(q_1v_1 + q_2v_2) = g(x)
\end{multline*}

We omit the coefficient off the $y''$ term since it can always be
divided out and it is simpler without it.

Re-aranging the above:
\begin{multline*}
	q_1(v_1'' + a_1v_1' + a_0v_1) + q_2(v_2'' + a_1v_2' + a_0v_2) \\
	+ q_1'v_1' + q_2'v_2' = g(x)
\end{multline*}

Since $v_1$ and $v_2$ are solutions to the homogeneous differential
equation, the first two terms are zero. This leaves:
\[
	q_1'v_1' + q_2'v_2' = g(x)
\]

By assumption we have $q_1'v_1 + q_2'v_2 = 0$ and so we now have two
equations with which we can work out $q_1$ and $q_2$.

\begin{ex}
	\[
		\frac{d^2y}{dx^2} + \frac{dy}{dx} - 2y = x
	\]

	The general solution of the homogeneous equation is
	\[
		y_c = Ae^{-2x} + Be^x
	\]
	So we have $v_1 = e^{-2x}$ and $v_2 = e^x$, with $g(x) = x$.

	Using variation of parameters, we know that
	$q_1'v_1' + q_2v_2' = g(x)$. We then have:
	\begin{align*}
		-2 q_1' e^{-2x} + q_2' e^x = x \\
		q_1' e^{-2x} + q_2' e^x = 0
	\end{align*}

	Solve for $q_1$:
	\begin{gather*}
		-3q_1'e^{-2x} = x \\
		\Rightarrow q_1' = -\frac{1}{3} xe^{2x} \\
		\int xe^{2x} = \frac{1}{2}xe^{2x} - \frac{1}{2} \int e^{2x} dx \\
		= \frac{1}{2}xe^{2x} - \frac{1}{4}e^{2x} + c \\
		\Rightarrow q_1 = -\frac{1}{6}e^{2x} (x - 1/2) + c
	\end{gather*}

	Solve for $q_2$:
	\begin{gather*}
		3q_2'e^x = x \\
		\Rightarrow q_2' = \frac{1}{3}xe^{-x} \\
		\int xe^{-x} dx = -xe_{-x} - \int -e^{-x} dx \\
		= -xe^{-x} - e{-x} +c \\
		\Rightarrow q_2 = -\frac{1}{3}e^{-x}(x+1) + c
	\end{gather*}

	So our particular solution is:
	\begin{align*}
		y_p &= -\frac{x}{6} + \frac{1}{12} + c_1e^{-2x}
		- \frac{x}{3} - \frac{1}{3} + c_2e^x \\
		&= -\frac{x}{2} - \frac{1}{4} + c_1e^{-2x} + c_2e^x
	\end{align*}

	Combining with the homogeneous solution, we get our general solution:
	\[
		y = Ae^{-2x} + Be^{x} - \frac{x}{2} - \frac{1}{4}
	\]
\end{ex}

\begin{ex}
	\[
		\frac{d^2y}{dx} + y = 1 + \tan x
	\]

	Since $g(x)$ is trigonometric, we choose our solution of the
	homogeneous in the form
	\[
		y_c = A\cos x + B\sin x
	\]

	\begin{align*}
		-q_1'\sin x + q_2'\cos x &= 1 + \tan x \\
		q_1'\cos x + q_2\sin x &= 0
	\end{align*}

	Solving for $q_2$:
	\begin{gather*}
		-q_1'\sin x \cos x + q_2'\cos^2x = \cos x + \sin x \\
		q_1' \sin x \cos x + q_2'\sin^2x = 0\\
		\Rightarrow q_2' = \cos x + \sin x \\
		\Rightarrow q_2 = \sin x - \cos x
	\end{gather*}

	Solving for $q_1$:
	\begin{gather*}
		q_1'\sin^2x - q_2'\cos x \sin x = -\sin x - \sin x \tan x \\
		q_1'\cos^2x + q_2'\cos x \sin x = 0 \\
		\Rightarrow q_1' = -\sin x - \sin x \tan x \\
		\Rightarrow q_1 = \cos x - \int \sin x \tan x dx
	\end{gather*}
\end{ex}

Note:
If equation is of form
\[
	y'' + a_1y' + a_0y = f(x) + p(x)
\]
Solve two separate equations
\begin{align*}
	y_1'' + a_1y_1' + a_0y_1 &= f(x) \\
	y_2'' + a_1y_2' + a_0y_2 &= p(x) \\
\end{align*}
and then $y_1 + y_2$ is a solution of the original equation.

\end{document}
