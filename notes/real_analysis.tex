\documentclass[a4paper,10pt]{article}
\usepackage{mystyle}

\begin{document}

\section{Limits}

\begin{defn}[Limit]
	$a$ is the limit of a sequence ${(xn)}_{n \in \NN}$ if $\forall \eps > 0 \quad \exists N \in \NN: n > N \implies | x_n -a | < \eps$.
	We write
	\[ \lim_{n \to \infty} x_n = a \]
\end{defn}

\begin{prop}
	Let $a$ be the limit of the sequence $x_n$, then $a$ is unique.
\end{prop}

\begin{proof}
	Assume we have two limits $a$ and $b$, then $\forall \eps > 0
	\quad \exists N_1, N_2 \in \NN$ such that
	\[ n > N_1 \implies |x_n - a| < \eps \]
	\[ n > N_2 \implies |x_n - b| < \eps \]

	Let $N = \max\{N_1, N_2\}$, then if $n > N$
	\[ |x_n - a| < \eps, \quad |x_n - b| < \eps \]

	\begin{align*}
		|b-a| &= |(x_n - a) + (b - x_n)| \leq |x_n - a| + |x_n - b| \\
		      &\implies |b-a| < 2\eps \\
		      &\implies b - a = 0 \implies a = b
	\end{align*}
\end{proof}

\begin{thm}[Archimedian Property]
	Let $x,y \in \RR, \, x>0$ then $\exists n \in \NN$ such that $xn > y$.
\end{thm}

\begin{proof}
	Assume $nx \leq y, \, \forall n \in \NN$.

	Let $S_x = \{ nx \mid n \in \NN \}$. By assumption, $S_x$ is
	bounded, and so by the completeness of the real numbers, we know
	the supremum of this set exists. $y$ is the maximum of the set
	so it must be the supremum too.

	$x > 0 \implies y-x < y$ so $y-x$ is not an upper bound for $S_x$.
	\begin{align*}
		&\implies \exists s \in S_x : y-x < s = kx, \, k \in \NN \\
		&\implies y < (k+1)x = mx \in S_x
	\end{align*}
	which is a contradiction to our assumption.
\end{proof}

\begin{cor}
	\[ \forall y \in \RR, \, \exists N \in \NN : n > y \]
\end{cor}

\begin{prop}
	Let $x_n = \frac{1}{n}, \, n \in \NN$, then $\lim_{n \to \infty}
	x_n = 0$.
\end{prop}

\begin{proof}
	Let $\frac{1}{\eps} \in \RR$. By the Archimedian Property,
	$\exists N \in \NN : N > \frac{1}{\eps}, \, \forall \eps > 0$.

	Let $n>N$, then
	\[ |x_n - 0| = \frac{1}{n} < \frac{1}{N} < \frac{1}{1/\eps} < \eps \]
\end{proof}

\begin{prop}
	Assume $\lim_{n \to \infty} x_n = a$ and $\lim_{n \to \infty}
	y_n = b$, then:
	\begin{enumerate}
		\item
			$\lim (x_n + y_n) = a + b$
		\item
			$\lim (x_n \dot y_n) = a \dot b$
		\item
			Assuming $y_n \neq 0 \, \forall n$ and $b \neq 0$,
			$\lim(\frac{x_n}{y_n}) = \frac{a}{b}$
		\item
			If a sequence is constant i.e. $x_n = c \,
			\forall n$, then $\lim x_n = c$
	\end{enumerate}
\end{prop}

\begin{proof}
	4:
	\begin{align*}
		&|x_n - c| = 0, \, \forall n \in \NN \\
		\implies &|x_n - c| < \eps, \, \forall \eps > 0 \in \RR, n \in \NN \\
		\implies &c = \lim x_n
	\end{align*}

	1:
	For a given $\eps > 0, \, \exists N_1, N_2 \in \NN$ such that
	\[ n > N_1 \implies |x_n - a| < \frac{\eps}{2} \]
	\[ n > N_2 \implies |y_n - b| < \frac{\eps}{2} \]
	\[
		n > \max \{ N_1, N_2 \} \implies | (x_n + y_n) - (a+b)|
	\]
	\[
		\leq |x_n - a| + |y_n - b|
		< \frac{\eps}{2} + \frac{\eps}{2} = \eps
	\]

	3:
	\[ \exists N_1 : |x_n - a| < |a|/2 \]
	\[ |a| = |a + x_n - x_n| \leq |a - x_n| + |x_n| < |a|/2 + |x_n| \]
	\[ \implies |a|/2 < |x_n| \implies 1/|x_n| < 2/|a| \]
	\[ \exists N_2 : |x_n -a| < \frac{|a|^2}{2}\eps \]
	\begin{align*}
		n > \max{N_1, N_2} \implies \\
		|\frac{1}{x_n} - \frac{1}{a}| &= |\frac{a-x_n}{ax_n} \\
		&= \frac{1}{|a|}\frac{1}{|x_n|}|x_n - a| \\
		&< \frac{1}{|a|}\frac{2}{|a|}|x_n - a| \\
		&< \frac{2}{|a|^2}\frac{a^2}{2}\eps = \eps
	\end{align*}

\end{proof}

\begin{ex}
	\[
		\lim \frac{n^2 - 1}{n_2 + 1}
		= \lim \frac{1 - \frac{1}{n^2}}{1 + \frac{1}{n^2}}
	\]
	\[ \lim \frac{1}{n} = 0 \implies \lim \frac{1}{n^2} = 0 \]
	\[ \implies \lim \frac{n^2 - 1}{n_2 + 1} = 1 \]
\end{ex}

\begin{defn}[Divergence]
	A sequence $x_n$ diverges to $+\infty$ if $\forall A > 0, \,
	\exists N \in \NN : n > N \implies x_n > A$. We say
	\[ \lim_{n \to \infty} x_n = \infty \]

	Similarly, if $\forall A < 0, \, \exists N \in \NN : n > N
	\implies x_n - A$, then
	\[ \lim_{n \to \infty} x_n = -\infty \]
\end{defn}

\begin{defn}[Increasing]
	A sequence is called increasing if $\forall n \in \NN, \, x_n
	\leq x_{n+1}$.
\end{defn}

\begin{defn}[Decreasing]
	A sequence is called decreasing if $\forall n \in \NN, \, x_n
	\geq x_{n+1}$.
\end{defn}

\begin{defn}[Monotonic]
	A sequence is monotonic if it is increasing or descreasing.
\end{defn}

\begin{defn}[Bounded]
	A sequence is bounded if $\exists A > 0 : |x_n| \leq A \,
	\forall n \in \NN$.
\end{defn}

\begin{thm}[MonontonicConvergence]
	If $x_n$ is bounded and increasing, it converges to $\sup\{x_1,
	\ldots, x_n\}$. If it is bounded and decreasing then it
	converges to $\inf\{x_1, \ldots, x_n\}$.
\end{thm}

\begin{proof}
	Assume the sequence is increasing.\\

	The supremum, $a$, of the sequence exists because $\RR$ is
	complete and the sequence is bounded. Take $\eps > 0$, then $a
	- \eps < a$ so $a - \eps$ is not a upper bound for $\{x_1,
	\ldots, x_n\}$ and there is $N \in \NN$ such that $x_N > a - \eps$.

	For every $n > N$ we have $x_n \geq x_N$, due to the increasing
	nature of the sequence. Hence,
	\[
		a - \eps < x_n \leq a < a + \eps
		\implies a - \eps < x_n < a + \eps
		\implies | x_n - a | < \eps
	\]
	and $\lim x_n = \sup\{x_1, \ldots, x_n\}$.\\

	Assume now that the sequence is decreasing.\\

	Again by completeness and boundedness $a = \inf \{x_1, \ldots,
	x_n\}$ is well defined.
	Fix $\eps > 0$ then $a + \eps > a$ and so $a + \eps$ is not a
	lower bound for $\{x_1, \ldots, x_n\}$ and there is $N \in \NN$
	such that $x_N < a + \eps$.
	Since $x_n$ is decreasing, $n > N \implies x_n \leq x_N$, so
	\[
		a - \eps < a \leq x_n < a + \eps
		\implies | x_n - a | < \eps
	\]
	and $\lim x_n = \inf \{x_1, \ldots, x_n\}$.
\end{proof}

\begin{ex}
	Let $x_1 = 2$ and $x_{n+1} = 2 + \sqrt{x_n}$, so $x_2 = 2 +
	\sqrt{2}$, $x_3 = 2 + \sqrt{2 + \sqrt{2}}$, etc.

	We aim to show that $2 \leq x_n \leq 4 \quad \forall n \in \NN$.

	$n=1$ is clearly true, so we may assume true for $n=k$.

	\[
		2 \leq x_k \leq 4 \implies \sqrt{2} \leq \sqrt{x_k} \leq \sqrt{4} = 2
	\]
	\[
		\implies 2 + \sqrt{2} \leq 2 + \sqrt{x_k} \leq 4
	\]
	\[
		2 + \sqrt{2} > 2 \implies x_{k+1} > 2 \implies 2 \leq x_{k+1} \leq 4
	\]

	So $x_n$ is bounded and increasing hence $\lim_{n \to \infty} x_n = 4$.
\end{ex}

\begin{defn}[Cauchy Sequence]
	A sequence is Cauchy if
	\[
		\forall \eps > 0 \, \exists N \in \NN :
		| x_n - x_m | < \eps \forall n,m > N
	\]
\end{defn}

\begin{thm}
	Any Cauchy sequence is bounded.
\end{thm}

\begin{proof}
	Let $x_n$ be a Cauchy sequence. Take $\eps = 1$, then $\exists
	N \in \NN : |x_n - x_m| < 1 \quad \forall n,m > N$.

	If $m = N+1$ then $|x_n - x_{N+1}| < 1 \iff x_{N+1} - 1 < x_n < x_{N+1} + 1$.

	$A = \max\{x_1, x_2, \ldots, x_N, x_{N+1}+1\}$
	$B = \min\{x_1, x_2, \ldots, x_N, x_{N+1}-1\}$

	So $B \leq x_n \leq A \quad \forall n$, hence $x_n$ is bounded.
\end{proof}

\begin{prop}
	If $S$ and $T$ are bounded subsets of $\RR$ and $S \subseteq T$,
	then $\inf T \leq \inf S$.
\end{prop}

\begin{proof}
	Let $t = \inf T$ and $s = \inf S$.

	Assume $s<t$. By completeness, $t$ cannot be a lower bound of
	$S$, hence
	\begin{align*}
		\exists \alpha \in S : \alpha < t \\
		&\implies \alpha < \beta \forall \beta \in T \\
		&\implies \alpha \notin T \\
		&\implies S \nsubseteq T
	\end{align*}

	The contrapositive statement has been proved, hence the original
	statement is true as well.
\end{proof}

\begin{thm}
	A sequence is Cauchy iff it converges.
\end{thm}

\begin{proof}
	Assume that a sequence converges such that $\lim x_n = a$, then
	\[
		\forall \eps > 0 \exists N \in \NN :
		| x_n - a | < \frac{\eps}{2}, \quad
		| x_m - a | < \frac{\eps}{2}, \quad
		\forall n,m > N
	\]
	\begin{align*}
		|x_n - x_m| &= |(x_n -a) - (x_m -a)| \\
			    &\leq |x_n -a| + |x_m -a| \\
			    &< \frac{\eps}{2} + \frac{\eps}{2} = \eps
	\end{align*}

	Assume now that $x_n$ is Cauchy, then we know it is bounded and
	by completeness, $\inf\{x_1, \ldots, x_n, x_{n+1}\}$ is well
	defined.

	Let
	\begin{align*}
		b_1 &= \inf \{x_1, \ldots\} \\
		b_2 &= \inf \{x_2, \ldots\} \\
		b_n &= \inf \{x_n, \ldots\}
	\end{align*}
	then
	\[
		b_1 \leq b_2 \leq \cdots \leq b_n \leq \cdots
	\]
	and
	\[
		b_n \leq x_n \leq B
	\]
	where $B$ is an upper bound $\forall n \in \NN$.

	By the Monotonic Covergence Theorem, $a = \lim b_n$ exists, so
	\[
		\forall \eps > 0 \exists N_1 \in \NN :
		|b_n - a| < \frac{\eps}{3}, \quad \forall n > N_1
	\]
	\[
		b_n + \frac{\eps}{3} > b_n = \inf\{x_n\}
	\]
	\[
		\implies \exists m_0 > N_1 :
		b_n < x_{m_0} < b_n + \frac{\eps}{3}
	\]

	$x_n$ is Cauchy, hence
	\[
		\exists N_2 \in \NN :
		|x_n - x_m| < \frac{\eps}{3}, \quad
		\forall n,m > N_2
	\]

	Let $N = \max\{N_1, N_2\}$, then
	\begin{align*}
		|x_n - a| &\leq |(x_n - x_m) + (x_m - b_n) + (b_n - a)| \\
			  &\leq |x_n - x_m| + |x_m - b_n| + |b_n - a| \\
			  &< \frac{\eps}{3} + \frac{\eps}{3}
				+ \frac{\eps}{3} = \eps
	\end{align*}
\end{proof}

\begin{ex}[Harmonic Series]
	\[
		y_n = 1 + \frac{1}{2} + \cdots + \frac{1}{n}
	\]
	We claim that $y_n$ diverges.
	\begin{align*}
		y_{2n} - y{n} &=
		\left(1 + \frac{1}{2} + \cdots + \frac{1}{2n} \right)
		\left(1 + \frac{1}{2} + \cdots + \frac{1}{n} \right) \\
		&= \frac{1}{n+1} + \cdots + \frac{1}{2n}
		> \frac{1}{2n} + \cdots + \frac{1}{2n} = \frac{1}{2}
	\end{align*}

	If we set $\eps = \frac{1}{3}$ then for $m = 2n$
	\[
		|y_m - y_n| > \eps
	\]
	hence the sequence is not Cauchy and so it cannot converge.
\end{ex}

\begin{cor}
	Any convergent sequence is bounded.
\end{cor}

\begin{ex}
	\[
		x_n = {(-1)}^n
	\]
	\[
		x_n = \pm 1 \quad \forall n \in \NN
	\]
	hence $x_n$ is bounded yet it does not converge.
\end{ex}

\section{Subsequences}

\begin{prop}
	If $x_n$ is a convergent sequence with $\lim_{n \to \infty} x_n
	= a$, then any subsequence $x_{n_k}$ coverges and $\lim_{k \to
	\infty} x_{n_k} = a$.
\end{prop}

\begin{proof}
	$x_n$ coverges so
	\[
		\forall \eps > 0 \exists N : |x_n - a| < \eps \quad \forall n > N
	\]

	Since $x_{n_k}$ is a subsequence of $x_n$, $n_k \geq k$ hence
	\[
		| x_{n_k} -a | < \eps \quad \forall k > N
		\implies \lim x_{n_k} = a
	\]
\end{proof}

\begin{thm}[Bolzano-Weierstrass]
	Any bounded sequence has a convergent subsequence.
\end{thm}

\begin{proof}
	Let $x_n$ be a bounded sequence, then
	\[
		\exists I_1 = [a,b] : x_n \in I_1, \, \forall n \geq 1
	\]

	If $c_1$ is the midpoint of $I_1$, then either $[a,c_1]$ or
	$[c_1,b]$ contains infinitely many terms of $x_n$. We call this
	$I_2$.

	We can continue in this manner, forming a sequnce $I_n$.

	We can form a subsequence of $x_n$ by taking $x_{n_1} \in I_1$,
	$x_{n_2} \in I_2$, etc.\ such that $n_1 < n_2 < \cdots < n_k$.

	Since the interval sizes are arbitrarily small, we can take
	$\eps = b_n - a_n$, where $I_n = [a_n,b_n]$. Hence
	\[
		|x-y| < b_n - a_n = \eps, \quad \forall x,y \in I_n
	\]
	and $x_{n_k}$ is Cauchy so it must converge.
\end{proof}

\section{Functions}

\begin{defn}[Continuous]
	A function $f: (a,b) \to \RR$ is continuous at $c \in (a,b)$
	if for any sequence of real numbers $x_n \in (a,b)$ where $\lim
	x_n = c$, one has $\lim f(x_n) = f(c)$.

	Equivalently, we could say $\lim_{x \to c} f(x) = f(c)$.

\end{defn}

\begin{defn}[Function Limits]
	Functions can have left and right limits.

	Let $f: (a,c) \cup (c,b) \to \RR$.

	Consider $x \in (a,c)$, then $\lim_{x \to c^-} f(x) = L$ is the
	left limit of $f$ if:
	\[
		\forall \eps > 0 \exists \delta > 0 :
		c - \delta < x < c \implies L - \eps < f(x) < L
	\]

	Now consider $x \in (c,b)$, then $\lim_{x \to c^+} f(x) = L$ is
	the right limit of $f$ if:
	\[
		\forall \eps > 0 \exists \delta > 0 :
		a < x < a + \delta \implies L < f(x) < L + \eps
	\]

	Let $x \in (a,c) \cup (c,b)$, then $\lim_{x \to c} f(x) = L$ is
	the limit of $f$ if:
	\[
		\forall \eps > 0 \exists \delta > 0 :
		|x-c| < \delta \implies |f(x) - L| < \eps
	\]

	If the limit exists, then so do the left and right limits and they are
	all equal. Conversely, if the left and right limits exist and are equal,
	the limit exists and share this common value.
\end{defn}

\begin{thm}
	Let $f$ be a function defined on $U = (a,c) \cup (c,b)$, then
	$\lim_{x \to c}f(x)$ exists iff for all sequences $x_n \in U$
	with $\lim_{x \to \infty} x_n = c$, we have
	\[
		\lim_{n \to \infty} f(x_n) = \lim_{x \to c} f(x)
	\]
\end{thm}

\begin{proof}
	Assume that $\lim_{x \to c}f(x) = L$.

	\[
		\forall \eps > 0 \, \exists \delta :
		|x-c| < \delta \implies |f(x)-L| < \eps
	\]

	Let $x_n$ be a sequence such that $\lim_{n \to\infty}x_n = c$, then
	\[
		\forall \eps' > 0 \, \exists N \in \NN :
		|x_n - c| < \eps', \, \forall n > N
	\]

	Let $\eps' = \delta$, then
	\begin{align*}
		|x_n - c| < \delta \quad \forall n > N \\
		&\implies |f(x_n) - L| < \eps \quad \forall n > N \\
		&\implies \lim f(x_n) = L
	\end{align*}

	Assume now that $\lim x_n = L$.

	Assume that $\lim f(x) \neq L$, then
	\[
		\exists \eps > 0 \, \forall \delta > 0 : \exists x :
		|x-c| < \delta \implies |f(x) - L| \geq \eps
	\]

	Fix $\eps$ and let $\delta = 1$, then there exists $x_1$ such that:
	\[
		|x_1 - c| < 1 \implies |f(x_1) - L| \geq \eps
	\]

	Let $\delta = \frac{1}{2}$, then there exists $x_2$ such that:
	\[
		|x_2 - c| < \frac{1}{2} \implies |f(x_2) - L| \geq \eps
	\]

	Continuing in this way, we obtain a sequence $x_n$ such that
	$\lim f(x_n) \neq L$, which is an obvious contradiction.
\end{proof}

\begin{rem}
	Let $f:[a,b] \to \RR$ be a function. $f$ is continuous as $c \in
	(a,b)$ if $\lim_{x \to c} f(x) = f(c)$.
	$f$ is continuous as $a \in (a,b)$ if $\lim_{x \to a^-} f(x) =
	f(a)$.
	$f$ is continuous as $a \in (a,b)$ if $\lim_{x \to b^+} f(x) =
	f(b)$.
\end{rem}

\begin{defn}[Continuous]
	$f:[a,b] \to \RR$ is continuous at $c$ if
	\[
		\forall \eps > 0 \, \exists \delta > 0 :
		|x-c| < \delta \implies |f(x)-f(c)| < \eps
	\]
	$f$ is continuous if it is continuous at every point in its
	domain.
\end{defn}

\subsection{Open and Closed Subsets}

\begin{defn}[Open Subset]
	A subset $U \subset \RR$ is called open if
	\[
		\forall a \in U \, \exists \eps > 0 :
		(a-\eps, a+\eps) \subset U
	\]
\end{defn}

\begin{defn}[Preimage]
	Let $f:[a,b] \to \RR$ and $U \subset \RR$. The preimage of $U$
	is defined as follows:
	\[
		f^{-1}(U) = \{ x \in [a,b] \mid f(x) \in U \}
	\]
\end{defn}

\begin{ex}
	\[
		f(x) = x^2, \quad f:\RR \to \RR, \quad U = \{0,1\}
	\]
	\[
		f^{-1}(U) = \{ x \in \RR \mid x^2 \in U \}
		= \{ \pm 1, 0 \}
	\]
\end{ex}

\begin{defn}[Closed Subset]
	A subset $V \in \RR$ is closed if its compliment is open.
\end{defn}

\begin{prop}
	$V \subset \RR$ is closed iff for any convergent sequence of
	real numbers $x_n$ in $V$, the limit is also in $V$.
\end{prop}

\begin{proof}
	Let $V = [a,b]$ and $x_n$ be a sequence in $V$.

	If every $x_n$ has a limit in $V$ then the sequence $a +
	\frac{\eps}{n}$, which clearly converges to $a$, must be in $V$.
	Likewise, the sequence $b - \frac{\eps}{n}$, whose limit is
	clearly $b$, must have it's limit in $V$.

	Assuming $V$ is closed, assume also that there is a sequence
	$x_n$ whose limit $L \notin [a,b]$, then $L = a -
	\frac{\eps}{2}$ or $b + \frac{\eps}{2}$ for some $\eps > 0 \in
	RR$.

	\begin{align*}
		L &= a - \frac{\eps}{2} \\
		  &\implies \exists N \in \NN : |x_n - a + \frac{\eps}{2}|
		< \frac{\eps}{2}, \,\forall n > N \\
		&\implies |x_n - a| - |- \frac{\eps}{2}| < \frac{\eps}{2} \\
		&\implies |x_n - a| < \eps \\
		&\implies a = \lim x_n
	\end{align*}
	This is a contradiction because $a \in V$.

	\begin{align*}
		L &= b + \frac{\eps}{2} \\
		  &\implies \exists N \in \NN : |x_n - b - \frac{\eps}{2}|
		< \frac{\eps}{2}, \,\forall n > N \\
		&\implies |x_n - b| - |\frac{\eps}{2}| < \frac{\eps}{2} \\
		&\implies |x_n - b| < \eps \\
		&\implies b = \lim x_n
	\end{align*}
	This is a contradiction because $b \in V$.
\end{proof}

\subsection{Properties of Open and Closed Sets}

\begin{itemize}
	\item A finite union of closed subsets is closed
	\item Any intersetion of closed subsets is closed
	\item Any union of open subsets is open
	\item A finite intersection of open subsets is open
\end{itemize}

\begin{proof}
	Number 2. \\

	Let $[a,b]$ and $[c,d]$ be two closed subsets in $\RR$, with a
	nonempty intersection.  If either is a subset of the other, then
	we are done.  If not, then either $c>a$ and $d>b$, or $a>c$ and
	$b>d$, resulting in the intersection being either $[c,b]$ or
	$[a,d]$, both of which are clearly closed.
\end{proof}

\begin{thm}
	For a function $f:(a,b) \to \RR$, $f$ is continuous iff the
	preimage of an open set under $f$ is also open.
\end{thm}

\begin{proof}
	Assume that $f$ is continuous. \\

	Let $U \subset \RR$ be open and let $W$ be the preimage of $U$
	under $f$.  Let $c \in W$ such that $f(c) = d \in U$. Given that
	$U$ is open, ther is $\eps > 0$ such that $(d-\eps, d+\eps)
	\subset U$.

	By the coninuity of $f$, there must be a $\delta > 0$ such that
	$|f(x) - f(c)| < \eps$ when $|x-c| < \delta$. We now need to
	show that $(c-\delta, c+\delta) \subset W$.

	\[
		z \in (c-\delta, c+\delta) \implies |z-c| < \delta
		\implies |f(z)-f(c)| < \eps
	\]
	\[
		\implies |f(z) - f(d)| < \eps
		\implies f(z) \in (d-\eps, d+\eps) \subset U
		\implies z \in W
	\]

	So, $(c-\delta, c+\delta) \subset W$, hence
	\[
		\forall c \in W \exists \delta > 0 :
		(c-\delta, c+\delta) \subset W
	\]
	and $W$ is open. \\

	Assume the openness of sets is preserved under $f$. \\

	Fix $\eps > 0$ and let $x \in (a,b)$ such that $f(x) = d \in
	RR$.  Let $U = (d-\eps, d+\eps) \subset \RR$, then $W =
	f^{-1}(U)$ is open by assumption.

	\[
		\implies \exists \delta > 0 :
		(x-\delta, x+\delta) \subset W
	\]

	Let $y \in (x-\delta, x+\delta)$, then $|x-y| < \delta$ hence
	$f(y) \in (d-\eps, d+\eps)$ and $|f(y) - f(x)| < \eps$.

\end{proof}


\subsection{Properties of Continuous Functions}

\begin{enumerate}
	\item The identity map is continuous
	\item
		If $f,g:(a,b) \to \RR$ are continuous, then $f+g$, $f-g$,
		$f\cdot g$ are all continuous. If $g(x) \neq 0 \forall x
		\in (a,b)$, $f(x)/g(x)$ is also continuous.
	\item $f \circ g$ is continuous
	\item $f(\lim x_n) = \lim f(x_n)$
\end{enumerate}

\begin{proof}
	\begin{enumerate}
		\item
			Fix $\eps$ and take $\delta = \eps$, then
			\[ |x-y| < \delta \implies |x-y| < \eps \implies |f(x)-f(y)| < \eps \]

		\item Follows from preservation of open sets.

		\item
			Let $c = \lim x_n$, then
			\[ \exists \eps > 0, \, N \in \NN : |x_n - c| < \eps, \, \forall n > N \]
			By continuity,
			\[ \exists \delta > 0 : |x_n -c| < \delta \implies |f(x)-f(c)| < \delta \]
			\[
				\implies |f(x_n) - f(c)| < \delta, \, \forall n > N
				\implies \lim f(x_n) = f(c) = f(\lim x_n)
			\]

		\item
			Follows from 4 and the properties of limits, since $f$ is continuous at $c$ iff
			$x_n \to c \implies f(x_n) \to f(c)$.
	\end{enumerate}
\end{proof}

\begin{cor}
	Any polynomial is a continuous function.
\end{cor}

\subsection{Intermediate Value Theorem}

\begin{thm}[Sandwich]
	Let $a_n$ and $b_n$ be sequences such that $\lim a_n = \lim b_n
	= C$. Then, if $c_n$ is a sequence such that $a_n \leq c_n \leq
	b_n \, \forall n \in \NN$, then $\lim c_n = C$.
\end{thm}

\begin{proof}
	$|c_n -C| \leq |a_n - C|$ or $|c_n -C| \leq |b_n - C|$, i.e.\
	$c_n$ is closer to $C$ than either $a_n$ or $b_n$.

	$\forall \eps > 0 \exists N_1,N_2 \in \NN : n > N_1 \implies
	|a_n - C| < \eps, \, n > N_2 \implies |b_n - C| < \eps$

	Let $N = \max(N_1, N_2)$, then for all $n > N$:
	\[ |a_n - C| < \eps \quad |b_n - C| < \eps \]
	\[ \implies |c_n -C| < \eps \implies \lim c_n = C \]
\end{proof}

\begin{thm}[Intermediate Value Theorem]
	Let $f:[a,b] \to \RR$ be a continuous function. If $f(a) = A
	\leq C \leq B = f(b)$, then $\exists c \in [a,b] : f(c) = C$.
\end{thm}

\begin{proof}
	Consider $g(x) = f(x) - C$. We aim to show that $g(c) = 0$ for some $c \in [a,b]$.

	Assume $A \leq C \leq B$, then
	\[ g(a) = f(a) - C \leq 0 \]
	\[ g(b) - f(b) - C \geq 0 \]

	If $g(a) = 0$ or $g(b) = 0$, the we are done. Assume $g(a) < 0$ and $g(b) > 0$.

	We can construct segments $[a_k, b_k] \subset [a,b]$, inductively as follows:
	\[ [a_1, b_1] = [a,b] \]
	Assume $[a_n, b_n]$ exists, and let $m$ be the midpoint of this interval.
	\[ m = \frac{b_n + a_n}{2} \in [a_n, b_n] \subset [a,b] \]
	Either $g(m) \leq 0$ or $g(m) > 0$. If $g(m) \leq 0$, then
	\[ [a_{n+1}, b_{n+1}] = [m, b_n] \]
	but if $g(m) > 0$, then
	\[ [a_{n+1}, b_{n+1}] = [a_n, m] \]

	We now have two bounded monotonic sequnces, $a_n$ and $b_n$, so
	they have well-defined limits.

	We also have
	\[
		b_{n+1} - a_{n+1} = \frac{1}{2^n}(b-a)
		\implies \lim a_n = \lim b_n = c
	\]

	Note that $g(a_k) \leq 0$ and $g(b_k) \geq 0$ for every $k \NN$.
	\[ \implies \lim g(a_n) = g(\lim a_n) = g(c) \leq 0 \]
	similarly,
	\[ \implies \lim g(b_n) = g(\lim b_n) = g(c) \geq 0 \]
	\[ \implies g(c) = 0 \implies f(c) = C \]
\end{proof}

\begin{defn}[Bounded Function]
	A function $f:I \to \RR$, where $I = [a,b] \subset \RR$, is bounded if
	\[ \exists M \in \RR, \, m > 0 : |f(x)| \leq M \, \forall x \in I. \]
	Equivalently, we could say
	\[ \exists A, B \in \RR : A \leq f(x) \leq B \, \forall x \in I. \]
\end{defn}

\begin{thm}
	If $f:[a,b] \to \RR$ is continuous, then $f$ is bounded.
\end{thm}

\begin{proof}
	Assume that $f$ is unbounded on $I = [a,b]$. Let $m = \frac{b+a}{2}$
	and divide $I$ into two subintervals: $[a,m]$ and $[m,b]$; $f$ must be
	unbounded on one of these intervals.

	Denote this subinterval by $[a_1, b_1]$, then construct a
	sequence of subintervals inductively.

	It is clear that $\lim a_n = \lim b_n = c$.

	By assumption, $f$ is unbounded in each of these intervals, in other words,
	\[ \exists c_n \in [a_n, b_n] : f(c_n) > N \forall N \in \NN. \]

	This implies $\lim f(c_n)$ does not exist, but this contradicts
	the sandwich theorem.
\end{proof}

\begin{thm}[Weierstrass Extremal Value Theorem]o
	Let $f:[a,b] \to \RR$ be a continuous function, then
	\[ \exists c,d \in [a,b] : f(c) \leq f(x) \leq f(d) \forall x \in [a,b] \]
\end{thm}

\begin{proof}
	$\{f(x) \mid x \in [a,b] \}$ is bounded, so $m = \inf \{f(x)\}$ exists.

	$m+\eps$ is not a lower bound, hence $\exists y : f(y) < m+\eps$.

	If $\eps = 1/n$, then denote with $y_n$ the corresponding $y$.

	\[
		m \leq f(y_n) < m + \frac{1}{n}
		\implies |f(y_n) -m| < \frac{1}{n} \, \forall n \in \NN
	\]
	\[ \implies \lim f(y_n) = m \]

	$y_n \in [a,b] \forall n \in \NN$, hence $y_n$ is bounded and
	has a convergent subsequence.

	\[
		\lim y_{n_k} = c \implies f(\lim y_{n_k}) = f(c)
		= \lim f(y_{n_k}) = m
	\]
	\[ \implies m = f(c) \implies f(x) \geq f(c) \forall x \in [a,b] \]

	Similarly, $M = \sup \{f(x)\}$, hence $M-\eps$ is not an upper bound and
	$\exists y \in (a,b) : M \geq f(y) > M-\eps$.
	\[ \implies |f(y) - M| < \eps \]

	Let $\eps = \frac{1}{n}$ for $y_n$, then $\lim f(y_n) = M$.
	Therefore there is a subsequence $y_{n_k}$ such that $\lim y_{n_k} = d$.

	\[ \implies f(\lim y_{n_k}) = f(d) = \lim f(y_{n_k}) = M \]
	\[ \implies f(d) = M \implies f(x) \leq f(d) \forall x \in [a,b] \]

	The extrema are reached at least once, simply because the limits
	have to exist within the bounded subsets in which their
	sequences are contained.
\end{proof}

\subsection{Applications of IVT}

\begin{itemize}
	\item If $f:[a,b] \to \RR$ is continuous, with $f(a) < 0$ and
		$f(b) > 0$, then $\exists c \in (a,b) : f(c) = 0$
	\item Let $f:[0,1] \to (0,1)$ be continuous, then
		$\exists c \in (0,1): f(c) = c$
\end{itemize}

\begin{proof}
	Let $g(x) = x - f(x)$, then $g(0) = -f(0) < 0$ and $g(1) = 1 - f(1) > 0$,
	since $f(x) \in (0,1)$.

	\[ \implies \exists c \in (0,1) : g(c) = 0 \implies f(c) = c \]
\end{proof}

\subsection{Differentiable Functions}

\begin{defn}[Differentiable]
	A function $f$ is differentiable at $c$ if the following limit exists:
	\[ \lim_{h \to 0} \frac{f(c+h) - f(c)}{h} \]
	$f$ is called differentiable if it is differentiable at any point in it's domain.
\end{defn}

\begin{lemma}
	$f$ differentiable at $c$ implies $f$ continuous at $c$.
\end{lemma}

\begin{proof}
	Let $f$ be differentialble at $c$.
	\[
		\implies f'(c) = \lim_{h \to 0} \frac{f(c+h) - f(c)}{h}
		= \lim_{x \to c} \frac{f(x) - f(c)}{x-c}
	\]
	\begin{align*}
		\lim_{x \to c} f(x) - f(c)
		&= \lim_{x \to c}(x-c) \lim_{x \to c} \frac{f(x) - f(c)}{x-c} \\
		&= 0
	\end{align*}
	\[
		\implies \lim_{x \to c} f(x) = f(c)
	\]
	hence $f$ is continuous at $c$.
\end{proof}

\begin{rem}
	Not all continuous functions are differentiable, e.g.\ $f(x) = |x|$.
\end{rem}

\begin{lemma}[Product Rule]
	$(f(x)g(x))' = f(x)g'(x) + f'(x)g(x)$.
\end{lemma}

\begin{proof}
	\begin{align*}
		&\lim_{h \to 0} \frac{f(x+h)g(x+h) - f(x)g(x)}{h} \\
		&= \lim_{h \to 0} \frac{f(x+h)g(x+h) - f(x+h)g(x) + f(x+h)g(x) - f(x)g(x)}{h} \\
		&= \lim_{h \to 0} \frac{f(x+h)(g(x+h) - g(x))}{h}
		+ \lim_{h \to 0} \frac{g(x)(f(x+h) - f(x))}{h} \\
		&= \lim_{h \to 0} f(x+h) \lim_{h \to 0} \frac{g(x+h) - g(x)}{h}
		+ g(x) \lim_{h \to 0} \frac{f(x+h) - f(x)}{h} \\
		&= f(x)g'(x) + g(x)f'(x)
	\end{align*}
\end{proof}

\begin{lemma}[Quotient Rule]
	\[ \left( \frac{f}{g} \right)' = \frac{f'g - g'f}{g^2} \]
\end{lemma}

\begin{proof}
	\begin{align*}
		&\lim_{h \to 0} \left(\frac{f(x+h)}{g(x+h)} - \frac{f(x)}{g(x)}\right)/h \\
		&= \lim_{h \to 0} \frac{f(x+h)g(x) - f(x)g(x+h)}{hg(x)g(x+h)} \\
		&= \lim_{h \to 0} \frac{f(x+h)g(x) - f(x)g(x) - f(x)g(x+h) + f(x)g(x)}{hg(x)g(x+h)} \\
		&= \lim_{h \to 0} \left(
			g(x)\frac{f(x+h) - f(x)}{h} - f(x)\frac{g(x+h) - g(x)}{h}
		\right)/g(x)g(x+h)
	\end{align*}
\end{proof}

\begin{lemma}
	$f$ is differentiable at $c$ iff $\exists A \in \RR$ and a continuous funtion $\alpha$ such that
	\[ f(x) = f(c) + A(x-c) + \alpha(x)(x-c) \]
	where $\lim_{x \to c} \alpha(x) = 0$.
	If this is so, then $A = f'(c)$.
\end{lemma}

\begin{proof}
	Assume $A$ and $\alpha$ exist.
	\[ f(c+h) = f(c) + Ah + \alpha(c+h)h \]
	\[
		\lim_{h \to 0} \frac{f(c+h) - f(c)}{h}
		= \lim_{h \to 0} \frac{Ah + h\alpha(c+h)}{h} = A
	\]
	\[ \implies f'(c) = A\]
	(since $\alpha(c) = 0$).

	Assume $f$ differentiable at $c$.

	Let $f'(c) = A = \lim_{h \to 0}\frac{f(c+h) - f(c)}{h}$.
	Let
	\[
		\alpha(x) =
		\begin{cases}
			\frac{f(x) - f(c)}{x-c} - A, &x \neq c \\
			0, &x = c
		\end{cases}
	\]
	\[ f(x) = f(c) + (x-c)A + (x-c)\alpha(x) \]
	Need to show that $\lim_{x \to c} \alpha(x) = 0$.

	\begin{align*}
		\lim_{x \to c} \alpha(x) &= \lim_{x \to c}\frac{f(x)-f(c)}{x-c} - f'(c) \\
		x = c+h &\implies \lim_{h \to 0} \frac{f(c+h)-f(c)}{h} - f'(c) \\
			&= 0
	\end{align*}
\end{proof}

\begin{lemma}[Chain Rule]
	Let $f:(a,b) \to (c,d) \subset \RR$ and $g:(c,d) \to \RR$, where
	$f$ is differentiable at $x_0 \in (a,b)$ and $g$ is
	differentiable at $y_0 = f(x_0)$. Then, $\phi(x) = g(f(x))$ is
	differentiable at $x_0$, and
	\[ \phi'(x_0) = g'(f(x_0))f'(x_0). \]
\end{lemma}


\begin{proof}
	Using the previous lemma, we know that:
	\[ f(x) = f(x_0) + A(x-x_0) + \alpha(x)(x-x_0) \]
	where
	\[ A = f'(x_0), \, \lim_{x \to x_0} \alpha(x) = 0 \]

	Similarly,
	\[ g(y) = g(y_0) + B(y-y_0) + \beta(y)(y-y_0) \]
	where
	\[ B = g'(y_0), \, \lim_{y \to y_0} \beta(y) = 0 \]

	Let $y = f(x)$ and $y_0 = f(x_0)$, then
	\[ g(f(x)) = g(f(x_0)) + B(f(x)-f(x_0)) + \beta(f(x))(f(x) - f(x_0)) \]
	\[
		\implies \phi(x) = \phi(x_0) + B(A(x-x_0) + (x-x_0)\alpha(x))
		+ \beta(f(x))(A(x-x_0) + (x-x_0)\alpha(x))
	\]
	\[ \phi(x_0) + BA(x-x_0) + \gamma(x)(x-x_0) \]
	where
	\[ \gamma(x) = B\alpha(x) + A\beta(f(x)) + \alpha(x)\beta(f(x)) \]

	$\lim_{x \to x_0} \gamma(x) = 0$ since $\lim_{x \to x_0} \alpha(x) = \lim_{x \to x_0} \beta(f(x)) = 0$.

	\[ \implies BA = \phi'(x_0) \]
	\[ \implies \phi'(x_0) = g'(f(x_0))f'(x_0) \]
\end{proof}

\begin{defn}[Local Maximum]
	Let $f:[a,b] \to \RR$. $x_0$ is a local maximum of $f$ if $\exists \delta > 0$ such that
	\[ f(x) \leq f(x_0) \forall x \in (x_0 - \delta, x_0 + \delta) \subset [a,b]. \]
\end{defn}

\begin{thm}[Fermat]
	Assume $f:[a,b] \to \RR$ is continuous, and differentiable at
	$x_0$. If $x_0$ is a local maximum, then $f'(x_0) = 0$.
\end{thm}

\begin{proof}
	Let $\delta > 0$ and $U = (x_0 - \delta, x_0 + \delta) \subset [a,b]$, where $x_0$ is a local max.

	Take $h > 0$ such that $x_0 + h \in U$, then
	\[
		\lim_{h \to 0} \frac{f(x_0 + h) - f(x_0)}{h} \leq 0
	\]

	We may also take $h < 0$ such that $x_0 + h \in U$, and
	\[
		\lim_{h \to 0} \frac{f(x_0 + h) - f(x_0)}{h} \geq 0
	\]

	\[ \implies f'(x_0) = 0 \]
\end{proof}

\begin{defn}[Critical Point]
	A point $x_0 \in (a,b)$ of $f:[a,b] \to \RR$ iff $f'(x_0) = 0$.
	Note that any local maxiumum or minimum is a critical point.
\end{defn}

\begin{cor}
	$\max f(x) = \max \{ f(A), f(B), f(x_1), \ldots, f(x_n) \}$, where
	$x_1, \ldots, x_n$ are critical points of $f$.
\end{cor}

\begin{ex}
	\[ f:[0,2] \to \RR, \quad f(x) = 1 + 5x - x^5 \]
	\begin{align*}
		f'(x) = 5 - 5x^4 &= 0 \\
		x^4 &= 1 \\
		x &= \pm 1
	\end{align*}

	$-1 \not \in [0,2]$ so $1$ is the only critical point we care about.

	We now check the values of the bounds and critical points:
	\[ f(1) = 1 \]
	\[ f(0) = 1 \]
	\[ f(2) = -21 \]
	hence, $\max f(x) = 1$ (on $[0,2]$).
\end{ex}

\begin{thm}[Rolle's Theorem]
	Let $f:[a,b] \to \RR$ be a continuous funtion, which is
	differentiable on $(a,b)$. If $f(a) = f(b)$, then there is $c
	\in (a,b)$ such that $f'(c) = 0$.
\end{thm}

\begin{proof}
	We know there are points $x_0$ and $x_1$ in $[a,b]$ such that
	$f(x_0) \leq f(x) \leq f(x_1) \forall x \in [a,b]$. (By taking
	the max and min of the endpoints and critical points).

	If $x_1$ and $x_0$ are the endpoints of the interval $[a,b]$,
	then the function must be constant, so $f'(x) = 0 \forall x \in
	[a,b]$.

	Assume that $x_1$ is within $(a,b)$, then $x_1$ is a local
	maximum and, by Fermat's Theorem, is a critical point. Hence
	$f'(x_1) = 0$.
\end{proof}

\begin{thm}[Mean Value Theorem (Lagrange)]
	Let $f:[a,b] \to \RR$ be a continuous function, which is
	differentiable on $(a,b)$, then:
	\[ \exists c \in (a,b) : f(b) - f(a) = f'(c)(b-a) \]
\end{thm}

\begin{proof}
	Consider a function
	\[ g(x) = f(x) - \frac{f(b) - f(a)}{b-a}x \]
	\begin{align*}
		g(b) - g(a) &= f(b) - \frac{f(b) - f(a)}{b-a}b
		- f(a) + \frac{f(b) - f(a)}{b-a}a \\
		&= \left( f(b) - f(a) \right)
		\left(1 - \frac{b}{b-a} + \frac{a}{b-a} \right) \\
		&= 0
	\end{align*}

	By Rolle's theorem, there is $c \in (a,b)$ such that $g'(c) = 0$.

	\[ \implies f'(c) = \frac{f(b) - f(a)}{b-a} \]
\end{proof}

\begin{cor}
	Let $f:[a,b] \to \RR$ be a continuous function, which is differentiable in $(a,b)$, then:
	\begin{enumerate}
		\item If $f'(x) > 0 \forall x \in (a,b)$, then f is strictly increasing.
		\item If $f'(x) < 0 \forall x \in (a,b)$, then f is strictly decreasing.
		\item If $f'(x) = 0 \forall x \in (a,b)$, then f is constant.
	\end{enumerate}
\end{cor}

\begin{proof}
	\begin{enumerate}
		\item Take $[c,d] \subset [a,b]$ and $x_0 \in (c,d)$, then by MVT we have
			\[ f(d) - f(c) = f'(x_0)(d-c) > 0 \implies f(d) > f(c) \]
		\item Similar to 1.
		\item Take $c,d \in [a,b]$, then $\exists x_0 \in (c,d)$ such that
			\[ f(d) - f(c) = f'(x_0)(d-c) = 0 \implies f(d) = f(c) \]
	\end{enumerate}
\end{proof}

\begin{ex}
	\[ f(x) = \frac{1}{3}x^3 - 3x^2 + 8x - 5 \]

	\[ f'(x) = x^2 - 6x + 8 = (x-4)(x-2) \]
	hence $x=4,2$ are critical points.

	\[ f'(3) = 9 - 18 + 8 = -1 \implies f'(x) < 0 \forall x \in (2,4) \]

	\[ f'(x) > 0 \forall x \in \RR \setminus [2,4] \]

	So $f(x)$ is increasing in $(2,4)$ and decreasing in $\RR \setminus [2,4]$.
\end{ex}

\begin{thm}[Cauchy MVT Generalisation]
	Assume $f,g:[a,b] \to \RR$ are continous and differentiable in $(a,b)$, then
	\[ \exists c \in (a,b) : (g(b) - g(a))f'(c) = (f(b) - f(a))g'(c) \]
\end{thm}

\begin{proof}
	If $g(b) = g(a)$ then by Rolle's Theorem, $\exists c \in (a,b) : g'(c) = 0$, and we are done.

	Assume $g(a) \neq g(b)$.

	Let $h(x) = (g(b) - g(a))f(x) - (f(b) - f(a))g(x)$.

	\begin{align*}
		h(b) - h(a) &= f(b)g(b) - f(b)g(a) - f(b)g(b) + f(a)g(b) \\
		&- f(a)g(b) + f(a)g(a) + f(b)g(a) - f(a)g(a) \\
		&= 0
	\end{align*}

	\[ \implies \exists c \in (a,b) : h'(c) = 0 \]

	\[ h'(x) = (g(b) - g(a))f'(x) - (f(b) - f(a))g'(x) \]
	\[ h'(c) = 0 \implies (g(b) - g(a))f'(c) = (f(b) - f(a))g'(c) \]
\end{proof}

\begin{rem}
	Setting $g$ to the identity map yields the standard MVT\@.
\end{rem}

\begin{lem}[L'hopital Rule]
	Assume $f$ and $g$ are differentiable in $(a,b)$ and $c \in (a,b) : f(c) = g(c) = 0$.
	Moreover, let $g(x), g'(x) \neq 0 \forall x \neq c$. Then
	\[ \lim_{x \to c} \frac{f(x)}{g(x)} = \lim_{x \to c} \frac{f'(x)}{g'(x)} \]
\end{lem}

\begin{proof}
	Let $x_n \in (c,b) : \lim_{n \to \infty} x_n = c$.

	By Cauchy MVT, $\exists y_n \in (c, x_n)$ such that
	\[ (g(x_n) - g(c))f'(y_n) = (f(x_n) - f(c))g'(y_n) \]

	By assumption, $f(c) = g(c) = 0$, so
	\[
		g(x_n)f'(y_n) = f(x_n)g'(y_n)
		\implies
		\frac{f'(y_n)}{g'(y_n)} = \frac{f(x_n)}{g(x_n)}
	\]

	\[ x_n \to c \implie y_n \to c \]

	\[
		\implies
		\lim_{n \to \infty}\frac{f(x_n)}{g(x_n)}
		= \lim_{n \to \infty}\frac{f'(y_n)}{g'(y_n)}
		= \lim_{x \to c^+}\frac{f'(x)}{g'(x)}
	\]

	We can use a similar method for $x_n \in (a,c)$, hence
	\[
		\lim_{x \to c^-}\frac{f(x)}{g(x)}
		= \lim_{x \to c^+}\frac{f(x)}{g(x)}
		= \lim_{x \to c}\frac{f'(x)}{g'(x)}
	\]
\end{proof}

\begin{ex}
	\[ \lim_{x \to 0} \frac{x - \sin x}{x^3} \]

	$x \to 0 \implies f(x), g(x) \to 0$, so we can apply l'hopital.

	\[
		\lim_{x \to 0} \frac{x - \sin x}{x^3}
		= \lim_{x \to 0} \frac{1 - \cos x}{3x^2}
	\]

	$x \to 0 \implies f(x), g(x) \to 0$, so we can apply l'hopital again.

	\[
		\lim_{x \to 0} \frac{1 - \cos x}{3x^2}
		= \lim_{x \to 0} \frac{\sin x}{6x}
		= \lim_{x \to 0} \frac{\cos x}{6}
		= \frac{1}{6}
	\]

\end{ex}

\section{Riemann Integration}

Denote by $B_{[a,b]}$ the set of all bounded functions on $[a,b]$, and
by $C_{[a,b]}$ the set of all continuous functions on $[a,b]$. Note that
$C_{[a,b]} \subset B_{[a,b]}$.

\begin{defn}[Partition]
	A partition of the interval $[a,b]$ is a (strictly increasing)
	sequence of numbers, $P := \{x_0, \ldots, x_n\}$, where $x_0 = a$
	and $x_n = b$.
\end{defn}

\begin{defn}[Partition width]
	The width of a partition $P$ is given
	\[ ||P|| = \max\{x_{i+1} - x_i\}.\]
\end{defn}

\begin{defn}[Refinement]
	If $P$ and $P'$ are partitions of $[a,b]$, then if $P \subset
	P'$, we say that $P'$ is a refinement of $P$.
\end{defn}

\begin{rem}
	If $P'$ is a refinement of a partition $P$, then $||P'|| \leq ||P||$.
\end{rem}

\begin{defn}[Riemann Sums]
	Let $f:[a,b] \to \RR$ be a bounded function and $P$ a partition of $[a,b]$.

	The lower Riemann sum of $f$ with respect to $P$ is
	\[ L(f,P) = \sum_{i=0}^{n-1} m_i (x_{i+1} - x_i) \]
	where $m_i = \inf_{x_i \leq x \leq x_{i+1}} \{f(x)\}$.

	The upper Riemann sum of $f$ with respect to $P$ is
	\[ U(f,P) = \sum_{i=0}^{n-1} M_i (x_{i+1} - x_i) \]
	where $M_i = \sup_{x_i \leq x \leq x_{i+1}} \{f(x)\}$.

	$m_i$ and $M_i$ are well defined as a result of the boundedness of $f$.
\end{defn}

\begin{rem}
	\[ L(f,P) \leq U(f,P) \]
\end{rem}

\begin{lemma}
	Assume $P'$ is a refinement of a partition $P$, then for a bounded function $f$
	\[ L(f,P) \leq L(f,P'), \quad U(f,P') \leq U(f,P). \]
\end{lemma}

\begin{proof}
	Consider $P = \{x_0, \ldots, x_n\}$ and $P' = \{x_0, x', x_1, \ldots, x_n\}$.

	\[ L(f,P) = m_0(x_1 - x_0) + \cdots + m_{n-1}(x_n - x_{n-1}) \]
	\[ L(f,P') = m_0'(x' - x_0) + m_0''(x_1 - x') + \cdots + m_{n-1}(x_n - x_{n-1}) \]

	It is clear that $m_0' \geq m_0$ and $m_0'' \geq m_0$, since $m_0 \in f([x_0, x'])$
	or $m_0 \in f([x', x_1])$ or bothe (if $m_0 = f(x')$).

	\begin{align*}
		L(f,P) - L(f,P') &= m_0(x_1 - x_0) - m_0(x' - x_0) - m_0''(x_1 - x') \\
				 &\leq m_0(x_1 - x_0 - x' + x_0 - x_1 + x') \\
		   		 &= 0
	\end{align*}
	\[ \implies L(f,P) \leq L(f,P') \]

	Similarly, $M_0'' \leq M_0$ and $M_0' \leq M_0$.
	\begin{align*}
		U(f,P) - U(f,P') &= M_0(x_1 - x_0) - M_0'(x' - x_0) + M_0''(x_1 - x') \\
				 &\geq 0
	\end{align*}
	\[ \implies U(f,P) \geq U(f,P') \]
\end{proof}


\end{document}
