\documentclass[a4paper,10pt]{article}
\usepackage{mystyle}

\begin{document}

\section{Limits}

\begin{defn}[Limit]
	$a$ is the limit of a sequence ${(xn)}_{n \in \NN}$ if $\forall \eps > 0 \quad \exists N \in \NN: n > N \implies | x_n -a | < \eps$.
	We write
	\[ \lim_{n \to \infty} x_n = a \]
\end{defn}

\begin{prop}
	Let $a$ be the limit of the sequence $x_n$, then $a$ is unique.
\end{prop}

\begin{proof}
	Assume we have two limits $a$ and $b$, then $\forall \eps > 0
	\quad \exists N_1, N_2 \in \NN$ such that
	\[ n > N_1 \implies |x_n - a| < \eps \]
	\[ n > N_2 \implies |x_n - b| < \eps \]

	Let $N = \max\{N_1, N_2\}$, then if $n > N$
	\[ |x_n - a| < \eps, \quad |x_n - b| < \eps \]

	\begin{align*}
		|b-a| &= |(x_n - a) + (b - x_n)| \leq |x_n - a| + |x_n - b| \\
		      &\implies |b-a| < 2\eps \\
		      &\implies b - a = 0 \implies a = b
	\end{align*}
\end{proof}

\begin{thm}[Archimedian Property]
	Let $x,y \in \RR, \, x>0$ then $\exists n \in \NN$ such that $xn > y$.
\end{thm}

\begin{proof}
	Assume $nx \leq y, \, \forall n \in \NN$.

	Let $S_x = \{ nx \mid n \in \NN \}$. By assumption, $S_x$ is
	bounded, and so by the completeness of the real numbers, we know
	the supremum of this set exists. $y$ is the maximum of the set
	so it must be the supremum too.

	 $x > 0 \implies y-x < y$ so $y-x$ is not an upper bound for $S_x$.
	 \begin{align*}
		 &\implies \exists s \in S_x : y-x < s = kx, \, k \in \NN \\
		 &\implies y < (k+1)x = mx \in S_x
	 \end{align*}
	 which is a contradiction to our assumption.
 \end{proof}

 \begin{cor}
	 \[ \forall y \in \RR, \, \exists N \in \NN : n > y \]
 \end{cor}

 \begin{prop}
	 Let $x_n = \frac{1}{n}, \, n \in \NN$, then $\lim_{n \to \infty}
	 x_n = 0$.
 \end{prop}

 \begin{proof}
	 Let $\frac{1}{\eps} \in \RR$. By the Archimedian Property,
	 $\exists N \in \NN : N > \frac{1}{\eps}, \, \forall \eps > 0$.

	 Let $n>N$, then
	 \[ |x_n - 0| = \frac{1}{n} < \frac{1}{N} < \frac{1}{1/\eps} < \eps \]
 \end{proof}

 \begin{prop}
	 Assume $\lim_{n \to \infty} x_n = a$ and $\lim_{n \to \infty}
	 y_n = b$, then:
	 \begin{enumerate}
		 \item
			 $\lim (x_n + y_n) = a + b$
		 \item
			 $\lim (x_n \dot y_n) = a \dot b$
		 \item
			 Assuming $y_n \neq 0 \, \forall n$ and $b \neq 0$,
			 $\lim(\frac{x_n}{y_n}) = \frac{a}{b}$
		 \item
			 If a sequence is constant i.e. $x_n = c \,
			 \forall n$, then $\lim x_n = c$
	 \end{enumerate}
 \end{prop}

 \begin{proof}
	 4:
	 \begin{align*}
		 &|x_n - c| = 0, \, \forall n \in \NN \\
		 \implies &|x_n - c| < \eps, \, \forall \eps > 0 \in \RR, n \in \NN \\
		 \implies &c = \lim x_n
	 \end{align*}

	 1:
	 For a given $\eps > 0, \, \exists N_1, N_2 \in \NN$ such that
	 \[ n > N_1 \implies |x_n - a| < \frac{\eps}{2} \]
	 \[ n > N_2 \implies |y_n - b| < \frac{\eps}{2} \]
	 \[
		 n > \max \{ N_1, N_2 \} \implies | (x_n + y_n) - (a+b)|
	 \]
	 \[
		 \leq |x_n - a| + |y_n - b|
		 < \frac{\eps}{2} + \frac{\eps}{2} = \eps
	 \]

	 3:
	 \[ \exists N_1 : |x_n - a| < |a|/2 \]
	 \[ |a| = |a + x_n - x_n| \leq |a - x_n| + |x_n| < |a|/2 + |x_n| \]
	 \[ \implies |a|/2 < |x_n| \implies 1/|x_n| < 2/|a| \]
	 \[ \exists N_2 : |x_n -a| < \frac{|a|^2}{2}\eps \]
	 \begin{align*}
		 n > \max{N_1, N_2} \implies \\
		 |\frac{1}{x_n} - \frac{1}{a}| &= |\frac{a-x_n}{ax_n} \\
		 &= \frac{1}{|a|}\frac{1}{|x_n|}|x_n - a| \\
		 &< \frac{1}{|a|}\frac{2}{|a|}|x_n - a| \\
		 &< \frac{2}{|a|^2}\frac{a^2}{2}\eps = \eps
	 \end{align*}

 \end{proof}

 \begin{ex}
	 \[
		 \lim \frac{n^2 - 1}{n_2 + 1}
		 = \lim \frac{1 - \frac{1}{n^2}}{1 + \frac{1}{n^2}}
	 \]
	 \[ \lim \frac{1}{n} = 0 \implies \lim \frac{1}{n^2} = 0 \]
	 \[ \implies \lim \frac{n^2 - 1}{n_2 + 1} = 1 \]
 \end{ex}

 \begin{defn}[Divergence]
	 A sequence $x_n$ diverges to $+\infty$ if $\forall A > 0, \,
	 \exists N \in \NN : n > N \implies x_n > A$. We say
	 \[ \lim_{n \to \infty} x_n = \infty \]

	 Similarly, if $\forall A < 0, \, \exists N \in \NN : n > N
	 \implies x_n - A$, then
	 \[ \lim_{n \to \infty} x_n = -\infty \]
 \end{defn}

 \begin{defn}[Increasing]
	 A sequence is called increasing if $\forall n \in \NN, \, x_n
	 \leq x_{n+1}$.
 \end{defn}

 \begin{defn}[Decreasing]
	 A sequence is called decreasing if $\forall n \in \NN, \, x_n
	 \geq x_{n+1}$.
 \end{defn}

 \begin{defn}[Monotonic]
	 A sequence is monotonic if it is increasing or descreasing.
 \end{defn}

 \begin{defn}[Bounded]
	 A sequence is bounded if $\exists A > 0 : |x_n| \leq A \,
	 \forall n \in \NN$.
 \end{defn}

 \begin{thm}[Monontonic Convergence]
	 If $x_n$ is bounded and increasing, it converges to $\sup\{x_1,
	 \ldots, x_n\}$. If it is bounded and decreasing then it
	 converges to $\inf\{x_1, \ldots, x_n\}$.
 \end{thm}

 \begin{proof}
	 Assume the sequence is increasing.\\

	 The supremum, $a$, of the sequence exists because $\RR$ is
	 complete and the sequence is bounded. Take $\eps > 0$, then $a
	 - \eps < a$ so $a - \eps$ is not a upper bound for $\{x_1,
	 \ldots, x_n\}$ and there is $N \in \NN$ such that $x_N > a - \eps$.

	 For every $n > N$ we have $x_n \geq x_N$, due to the increasing
	 nature of the sequence. Hence,
	 \[
		 a - \eps < x_n \leq a < a + \eps
		 \implies a - \eps < x_n < a + \eps
		 \implies | x_n - a | < \eps
	 \]
	 and $\lim x_n = \sup\{x_1, \ldots, x_n\}$.\\

	 Assume now that the sequence is decreasing.\\

	 Again by completeness and boundedness $a = \inf \{x_1, \ldots,
	 x_n\}$ is well defined.
	 Fix $\eps > 0$ then $a + \eps > a$ and so $a + \eps$ is not a
	 lower bound for $\{x_1, \ldots, x_n\}$ and there is $N \in \NN$
	 such that $x_N < a + \eps$.
	 Since $x_n$ is decreasing, $n > N \implies x_n \leq x_N$, so
	 \[
		 a - \eps < a \leq x_n < a + \eps
		 \implies | x_n - a | < \eps
	 \]
	 and $\lim x_n = \inf \{x_1, \ldots, x_n\}$.
 \end{proof}

\end{document}
