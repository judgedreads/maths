\documentclass[a4paper,10pt]{article}
\usepackage{mystyle}

\begin{document}

\section{Limits}

\begin{defn}[Limit]
	$a$ is the limit of a sequence ${(xn)}_{n \in \NN}$ if $\forall \eps > 0 \quad \exists N \in \NN: n > N \implies | x_n -a | < \eps$.
	We write
	\[ \lim_{n \to \infty} x_n = a \]
\end{defn}

\begin{prop}
	Let $a$ be the limit of the sequence $x_n$, then $a$ is unique.
\end{prop}

\begin{proof}
	Assume we have two limits $a$ and $b$, then $\forall \eps > 0
	\quad \exists N_1, N_2 \in \NN$ such that
	\[ n > N_1 \implies |x_n - a| < \eps \]
	\[ n > N_2 \implies |x_n - b| < \eps \]

	Let $N = \max\{N_1, N_2\}$, then if $n > N$
	\[ |x_n - a| < \eps, \quad |x_n - b| < \eps \]

	\begin{align*}
		|b-a| &= |(x_n - a) + (b - x_n)| \leq |x_n - a| + |x_n - b| \\
		      &\implies |b-a| < 2\eps \\
		      &\implies b - a = 0 \implies a = b
	\end{align*}
\end{proof}

\begin{thm}[Archimedian Property]
	Let $x,y \in \RR, \, x>0$ then $\exists n \in \NN$ such that $xn > y$.
\end{thm}

\begin{proof}
	Assume $nx \leq y, \, \forall n \in \NN$.

	Let $S_x = \{ nx \mid n \in \NN \}$. By assumption, $S_x$ is
	bounded, and so by the completeness of the real numbers, we know
	the supremum of this set exists. $y$ is the maximum of the set
	so it must be the supremum too.

	$x > 0 \implies y-x < y$ so $y-x$ is not an upper bound for $S_x$.
	\begin{align*}
		&\implies \exists s \in S_x : y-x < s = kx, \, k \in \NN \\
		&\implies y < (k+1)x = mx \in S_x
	\end{align*}
	which is a contradiction to our assumption.
\end{proof}

\begin{cor}
	\[ \forall y \in \RR, \, \exists N \in \NN : n > y \]
\end{cor}

\begin{prop}
	Let $x_n = \frac{1}{n}, \, n \in \NN$, then $\lim_{n \to \infty}
	x_n = 0$.
\end{prop}

\begin{proof}
	Let $\frac{1}{\eps} \in \RR$. By the Archimedian Property,
	$\exists N \in \NN : N > \frac{1}{\eps}, \, \forall \eps > 0$.

	Let $n>N$, then
	\[ |x_n - 0| = \frac{1}{n} < \frac{1}{N} < \frac{1}{1/\eps} < \eps \]
\end{proof}

\begin{prop}
	Assume $\lim_{n \to \infty} x_n = a$ and $\lim_{n \to \infty}
	y_n = b$, then:
	\begin{enumerate}
		\item
			$\lim (x_n + y_n) = a + b$
		\item
			$\lim (x_n \dot y_n) = a \dot b$
		\item
			Assuming $y_n \neq 0 \, \forall n$ and $b \neq 0$,
			$\lim(\frac{x_n}{y_n}) = \frac{a}{b}$
		\item
			If a sequence is constant i.e. $x_n = c \,
			\forall n$, then $\lim x_n = c$
	\end{enumerate}
\end{prop}

\begin{proof}
	4:
	\begin{align*}
		&|x_n - c| = 0, \, \forall n \in \NN \\
		\implies &|x_n - c| < \eps, \, \forall \eps > 0 \in \RR, n \in \NN \\
		\implies &c = \lim x_n
	\end{align*}

	1:
	For a given $\eps > 0, \, \exists N_1, N_2 \in \NN$ such that
	\[ n > N_1 \implies |x_n - a| < \frac{\eps}{2} \]
	\[ n > N_2 \implies |y_n - b| < \frac{\eps}{2} \]
	\[
		n > \max \{ N_1, N_2 \} \implies | (x_n + y_n) - (a+b)|
	\]
	\[
		\leq |x_n - a| + |y_n - b|
		< \frac{\eps}{2} + \frac{\eps}{2} = \eps
	\]

	3:
	\[ \exists N_1 : |x_n - a| < |a|/2 \]
	\[ |a| = |a + x_n - x_n| \leq |a - x_n| + |x_n| < |a|/2 + |x_n| \]
	\[ \implies |a|/2 < |x_n| \implies 1/|x_n| < 2/|a| \]
	\[ \exists N_2 : |x_n -a| < \frac{|a|^2}{2}\eps \]
	\begin{align*}
		n > \max{N_1, N_2} \implies \\
		|\frac{1}{x_n} - \frac{1}{a}| &= |\frac{a-x_n}{ax_n} \\
		&= \frac{1}{|a|}\frac{1}{|x_n|}|x_n - a| \\
		&< \frac{1}{|a|}\frac{2}{|a|}|x_n - a| \\
		&< \frac{2}{|a|^2}\frac{a^2}{2}\eps = \eps
	\end{align*}

\end{proof}

\begin{ex}
	\[
		\lim \frac{n^2 - 1}{n_2 + 1}
		= \lim \frac{1 - \frac{1}{n^2}}{1 + \frac{1}{n^2}}
	\]
	\[ \lim \frac{1}{n} = 0 \implies \lim \frac{1}{n^2} = 0 \]
	\[ \implies \lim \frac{n^2 - 1}{n_2 + 1} = 1 \]
\end{ex}

\begin{defn}[Divergence]
	A sequence $x_n$ diverges to $+\infty$ if $\forall A > 0, \,
	\exists N \in \NN : n > N \implies x_n > A$. We say
	\[ \lim_{n \to \infty} x_n = \infty \]

	Similarly, if $\forall A < 0, \, \exists N \in \NN : n > N
	\implies x_n - A$, then
	\[ \lim_{n \to \infty} x_n = -\infty \]
\end{defn}

\begin{defn}[Increasing]
	A sequence is called increasing if $\forall n \in \NN, \, x_n
	\leq x_{n+1}$.
\end{defn}

\begin{defn}[Decreasing]
	A sequence is called decreasing if $\forall n \in \NN, \, x_n
	\geq x_{n+1}$.
\end{defn}

\begin{defn}[Monotonic]
	A sequence is monotonic if it is increasing or descreasing.
\end{defn}

\begin{defn}[Bounded]
	A sequence is bounded if $\exists A > 0 : |x_n| \leq A \,
	\forall n \in \NN$.
\end{defn}

\begin{thm}[MonontonicConvergence]
	If $x_n$ is bounded and increasing, it converges to $\sup\{x_1,
	\ldots, x_n\}$. If it is bounded and decreasing then it
	converges to $\inf\{x_1, \ldots, x_n\}$.
\end{thm}

\begin{proof}
	Assume the sequence is increasing.\\

	The supremum, $a$, of the sequence exists because $\RR$ is
	complete and the sequence is bounded. Take $\eps > 0$, then $a
	- \eps < a$ so $a - \eps$ is not a upper bound for $\{x_1,
	\ldots, x_n\}$ and there is $N \in \NN$ such that $x_N > a - \eps$.

	For every $n > N$ we have $x_n \geq x_N$, due to the increasing
	nature of the sequence. Hence,
	\[
		a - \eps < x_n \leq a < a + \eps
		\implies a - \eps < x_n < a + \eps
		\implies | x_n - a | < \eps
	\]
	and $\lim x_n = \sup\{x_1, \ldots, x_n\}$.\\

	Assume now that the sequence is decreasing.\\

	Again by completeness and boundedness $a = \inf \{x_1, \ldots,
	x_n\}$ is well defined.
	Fix $\eps > 0$ then $a + \eps > a$ and so $a + \eps$ is not a
	lower bound for $\{x_1, \ldots, x_n\}$ and there is $N \in \NN$
	such that $x_N < a + \eps$.
	Since $x_n$ is decreasing, $n > N \implies x_n \leq x_N$, so
	\[
		a - \eps < a \leq x_n < a + \eps
		\implies | x_n - a | < \eps
	\]
	and $\lim x_n = \inf \{x_1, \ldots, x_n\}$.
\end{proof}

\begin{ex}
	Let $x_1 = 2$ and $x_{n+1} = 2 + \sqrt{x_n}$, so $x_2 = 2 +
	\sqrt{2}$, $x_3 = 2 + \sqrt{2 + \sqrt{2}}$, etc.

	We aim to show that $2 \leq x_n \leq 4 \quad \forall n \in \NN$.

	$n=1$ is clearly true, so we may assume true for $n=k$.

	\[
		2 \leq x_k \leq 4 \implies \sqrt{2} \leq \sqrt{x_k} \leq \sqrt{4} = 2
	\]
	\[
		\implies 2 + \sqrt{2} \leq 2 + \sqrt{x_k} \leq 4
	\]
	\[
		2 + \sqrt{2} > 2 \implies x_{k+1} > 2 \implies 2 \leq x_{k+1} \leq 4
	\]

	So $x_n$ is bounded and increasing hence $\lim_{n \to \infty} x_n = 4$.
\end{ex}

\begin{defn}[Cauchy Sequence]
	A sequence is Cauchy if
	\[
		\forall \eps > 0 \, \exists N \in \NN :
		| x_n - x_m | < \eps \forall n,m > N
	\]
\end{defn}

\begin{thm}
	Any Cauchy sequence is bounded.
\end{thm}

\begin{proof}
	Let $x_n$ be a Cauchy sequence. Take $\eps = 1$, then $\exists
	N \in \NN : |x_n - x_m| < 1 \quad \forall n,m > N$.

	If $m = N+1$ then $|x_n - x_{N+1}| < 1 \iff x_{N+1} - 1 < x_n < x_{N+1} + 1$.

	$A = \max\{x_1, x_2, \ldots, x_N, x_{N+1}+1\}$
	$B = \min\{x_1, x_2, \ldots, x_N, x_{N+1}-1\}$

	So $B \leq x_n \leq A \quad \forall n$, hence $x_n$ is bounded.
\end{proof}

\begin{prop}
	If $S$ and $T$ are bounded subsets of $\RR$ and $S \subseteq T$,
	then $\inf T \leq \inf S$.
\end{prop}

\begin{proof}
	Let $t = \inf T$ and $s = \inf S$.

	Assume $s<t$. By completeness, $t$ cannot be a lower bound of
	$S$, hence
	\begin{align*}
		\exists \alpha \in S : \alpha < t \\
		&\implies \alpha < \beta \forall \beta \in T \\
		&\implies \alpha \notin T \\
		&\implies S \nsubseteq T
	\end{align*}

	The contrapositive statement has been proved, hence the original
	statement is true as well.
\end{proof}

\begin{thm}
	A sequence is Cauchy iff it converges.
\end{thm}

\begin{proof}
	Assume that a sequence converges such that $\lim x_n = a$, then
	\[
		\forall \eps > 0 \exists N \in \NN :
		| x_n - a | < \frac{\eps}{2}, \quad
		| x_m - a | < \frac{\eps}{2}, \quad
		\forall n,m > N
	\]
	\begin{align*}
		|x_n - x_m| &= |(x_n -a) - (x_m -a)| \\
			    &\leq |x_n -a| + |x_m -a| \\
			    &< \frac{\eps}{2} + \frac{\eps}{2} = \eps
	\end{align*}

	Assume now that $x_n$ is Cauchy, then we know it is bounded and
	by completeness, $\inf\{x_1, \ldots, x_n, x_{n+1}\}$ is well
	defined.

	Let
	\begin{align*}
		b_1 &= \inf \{x_1, \ldots\} \\
		b_2 &= \inf \{x_2, \ldots\} \\
		b_n &= \inf \{x_n, \ldots\}
	\end{align*}
	then
	\[
		b_1 \leq b_2 \leq \cdots \leq b_n \leq \cdots
	\]
	and
	\[
		b_n \leq x_n \leq B
	\]
	where $B$ is an upper bound $\forall n \in \NN$.

	By the Monotonic Covergence Theorem, $a = \lim b_n$ exists, so
	\[
		\forall \eps > 0 \exists N_1 \in \NN :
		|b_n - a| < \frac{\eps}{3}, \quad \forall n > N_1
	\]
	\[
		b_n + \frac{\eps}{3} > b_n = \inf\{x_n\}
	\]
	\[
		\implies \exists m_0 > N_1 :
		b_n < x_{m_0} < b_n + \frac{\eps}{3}
	\]

	$x_n$ is Cauchy, hence
	\[
		\exists N_2 \in \NN :
		|x_n - x_m| < \frac{\eps}{3}, \quad
		\forall n,m > N_2
	\]

	Let $N = \max\{N_1, N_2\}$, then
	\begin{align*}
		|x_n - a| &\leq |(x_n - x_m) + (x_m - b_n) + (b_n - a)| \\
			  &\leq |x_n - x_m| + |x_m - b_n| + |b_n - a| \\
			  &< \frac{\eps}{3} + \frac{\eps}{3}
				+ \frac{\eps}{3} = \eps
	\end{align*}
\end{proof}

\begin{ex}[Harmonic Series]
	\[
		y_n = 1 + \frac{1}{2} + \cdots + \frac{1}{n}
	\]
	We claim that $y_n$ diverges.
	\begin{align*}
		y_{2n} - y{n} &=
		\left(1 + \frac{1}{2} + \cdots + \frac{1}{2n} \right)
		\left(1 + \frac{1}{2} + \cdots + \frac{1}{n} \right) \\
		&= \frac{1}{n+1} + \cdots + \frac{1}{2n}
		> \frac{1}{2n} + \cdots + \frac{1}{2n} = \frac{1}{2}
	\end{align*}

	If we set $\eps = \frac{1}{3}$ then for $m = 2n$
	\[
		|y_m - y_n| > \eps
	\]
	hence the sequence is not Cauchy and so it cannot converge.
\end{ex}

\begin{cor}
	Any convergent sequence is bounded.
\end{cor}

\begin{ex}
	\[
		x_n = {(-1)}^n
	\]
	\[
		x_n = \pm 1 \quad \forall n \in \NN
	\]
	hence $x_n$ is bounded yet it does not converge.
\end{ex}

\section{Subsequences}

\begin{prop}
	If $x_n$ is a convergent sequence with $\lim_{n \to \infty} x_n
	= a$, then any subsequence $x_{n_k}$ coverges and $\lim_{k \to
	\infty} x_{n_k} = a$.
\end{prop}

\begin{proof}
	$x_n$ coverges so
	\[
		\forall \eps > 0 \exists N : |x_n - a| < \eps \quad \forall n > N
	\]

	Since $x_{n_k}$ is a subsequence of $x_n$, $n_k \geq k$ hence
	\[
		| x_{n_k} -a | < \eps \quad \forall k > N
		\implies \lim x_{n_k} = a
	\]
\end{proof}

\begin{thm}[Bolzano-Weierstrass]
	Any bounded sequence has a convergent subsequence.
\end{thm}

\begin{proof}
	Let $x_n$ be a bounded sequence, then
	\[
		\exists I_1 = [a,b] : x_n \in I_1, \, \forall n \geq 1
	\]

	If $c_1$ is the midpoint of $I_1$, then either $[a,c_1]$ or
	$[c_1,b]$ contains infinitely many terms of $x_n$. We call this
	$I_2$.

	We can continue in this manner, forming a sequnce $I_n$.

	We can form a subsequence of $x_n$ by taking $x_{n_1} \in I_1$,
	$x_{n_2} \in I_2$, etc.\ such that $n_1 < n_2 < \cdots < n_k$.

	Since the interval sizes are arbitrarily small, we can take
	$\eps = b_n - a_n$, where $I_n = [a_n,b_n]$. Hence
	\[
		|x-y| < b_n - a_n = \eps, \quad \forall x,y \in I_n
	\]
	and $x_{n_k}$ is Cauchy so it must converge.
\end{proof}

\section{Functions}

\begin{defn}[Continuous]
	A function $f: (a,b) \to \RR$ is continuous at $c \in (a,b)$
	if for any sequence of real numbers $x_n \in (a,b)$ where $\lim
	x_n = c$, one has $\lim f(x_n) = f(c)$.

	Equivalently, we could say $\lim_{x \to c} f(x) = f(c)$.

\end{defn}

\begin{defn}[Function Limits]
	Functions can have left and right limits.

	Let $f: (a,c) \cup (c,b) \to \RR$.

	Consider $x \in (a,c)$, then $\lim_{x \to c^-} f(x) = L$ is the
	left limit of $f$ if:
	\[
		\forall \eps > 0 \exists \delta > 0 :
		c - \delta < x < c \implies L - \eps < f(x) < L
	\]

	Now consider $x \in (c,b)$, then $\lim_{x \to c^+} f(x) = L$ is
	the right limit of $f$ if:
	\[
		\forall \eps > 0 \exists \delta > 0 :
		a < x < a + \delta \implies L < f(x) < L + \eps
	\]

	Let $x \in (a,c) \cup (c,b)$, then $\lim_{x \to c} f(x) = L$ is
	the limit of $f$ if:
	\[
		\forall \eps > 0 \exists \delta > 0 :
		|x-c| < \delta \implies |f(x) - L| < \eps
	\]

	If the limit exists, then so do the left and right limits and they are
	all equal. Conversely, if the left and right limits exist and are equal,
	the limit exists and share this common value.
\end{defn}


\end{document}
